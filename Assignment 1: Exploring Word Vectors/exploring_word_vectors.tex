
% Default to the notebook output style

    


% Inherit from the specified cell style.




    
\documentclass[11pt]{article}

    
    
    \usepackage[T1]{fontenc}
    % Nicer default font (+ math font) than Computer Modern for most use cases
    \usepackage{mathpazo}

    % Basic figure setup, for now with no caption control since it's done
    % automatically by Pandoc (which extracts ![](path) syntax from Markdown).
    \usepackage{graphicx}
    % We will generate all images so they have a width \maxwidth. This means
    % that they will get their normal width if they fit onto the page, but
    % are scaled down if they would overflow the margins.
    \makeatletter
    \def\maxwidth{\ifdim\Gin@nat@width>\linewidth\linewidth
    \else\Gin@nat@width\fi}
    \makeatother
    \let\Oldincludegraphics\includegraphics
    % Set max figure width to be 80% of text width, for now hardcoded.
    \renewcommand{\includegraphics}[1]{\Oldincludegraphics[width=.8\maxwidth]{#1}}
    % Ensure that by default, figures have no caption (until we provide a
    % proper Figure object with a Caption API and a way to capture that
    % in the conversion process - todo).
    \usepackage{caption}
    \DeclareCaptionLabelFormat{nolabel}{}
    \captionsetup{labelformat=nolabel}

    \usepackage{adjustbox} % Used to constrain images to a maximum size 
    \usepackage{xcolor} % Allow colors to be defined
    \usepackage{enumerate} % Needed for markdown enumerations to work
    \usepackage{geometry} % Used to adjust the document margins
    \usepackage{amsmath} % Equations
    \usepackage{amssymb} % Equations
    \usepackage{textcomp} % defines textquotesingle
    % Hack from http://tex.stackexchange.com/a/47451/13684:
    \AtBeginDocument{%
        \def\PYZsq{\textquotesingle}% Upright quotes in Pygmentized code
    }
    \usepackage{upquote} % Upright quotes for verbatim code
    \usepackage{eurosym} % defines \euro
    \usepackage[mathletters]{ucs} % Extended unicode (utf-8) support
    \usepackage[utf8x]{inputenc} % Allow utf-8 characters in the tex document
    \usepackage{fancyvrb} % verbatim replacement that allows latex
    \usepackage{grffile} % extends the file name processing of package graphics 
                         % to support a larger range 
    % The hyperref package gives us a pdf with properly built
    % internal navigation ('pdf bookmarks' for the table of contents,
    % internal cross-reference links, web links for URLs, etc.)
    \usepackage{hyperref}
    \usepackage{longtable} % longtable support required by pandoc >1.10
    \usepackage{booktabs}  % table support for pandoc > 1.12.2
    \usepackage[inline]{enumitem} % IRkernel/repr support (it uses the enumerate* environment)
    \usepackage[normalem]{ulem} % ulem is needed to support strikethroughs (\sout)
                                % normalem makes italics be italics, not underlines
    \usepackage{mathrsfs}
    

    
    
    % Colors for the hyperref package
    \definecolor{urlcolor}{rgb}{0,.145,.698}
    \definecolor{linkcolor}{rgb}{.71,0.21,0.01}
    \definecolor{citecolor}{rgb}{.12,.54,.11}

    % ANSI colors
    \definecolor{ansi-black}{HTML}{3E424D}
    \definecolor{ansi-black-intense}{HTML}{282C36}
    \definecolor{ansi-red}{HTML}{E75C58}
    \definecolor{ansi-red-intense}{HTML}{B22B31}
    \definecolor{ansi-green}{HTML}{00A250}
    \definecolor{ansi-green-intense}{HTML}{007427}
    \definecolor{ansi-yellow}{HTML}{DDB62B}
    \definecolor{ansi-yellow-intense}{HTML}{B27D12}
    \definecolor{ansi-blue}{HTML}{208FFB}
    \definecolor{ansi-blue-intense}{HTML}{0065CA}
    \definecolor{ansi-magenta}{HTML}{D160C4}
    \definecolor{ansi-magenta-intense}{HTML}{A03196}
    \definecolor{ansi-cyan}{HTML}{60C6C8}
    \definecolor{ansi-cyan-intense}{HTML}{258F8F}
    \definecolor{ansi-white}{HTML}{C5C1B4}
    \definecolor{ansi-white-intense}{HTML}{A1A6B2}
    \definecolor{ansi-default-inverse-fg}{HTML}{FFFFFF}
    \definecolor{ansi-default-inverse-bg}{HTML}{000000}

    % commands and environments needed by pandoc snippets
    % extracted from the output of `pandoc -s`
    \providecommand{\tightlist}{%
      \setlength{\itemsep}{0pt}\setlength{\parskip}{0pt}}
    \DefineVerbatimEnvironment{Highlighting}{Verbatim}{commandchars=\\\{\}}
    % Add ',fontsize=\small' for more characters per line
    \newenvironment{Shaded}{}{}
    \newcommand{\KeywordTok}[1]{\textcolor[rgb]{0.00,0.44,0.13}{\textbf{{#1}}}}
    \newcommand{\DataTypeTok}[1]{\textcolor[rgb]{0.56,0.13,0.00}{{#1}}}
    \newcommand{\DecValTok}[1]{\textcolor[rgb]{0.25,0.63,0.44}{{#1}}}
    \newcommand{\BaseNTok}[1]{\textcolor[rgb]{0.25,0.63,0.44}{{#1}}}
    \newcommand{\FloatTok}[1]{\textcolor[rgb]{0.25,0.63,0.44}{{#1}}}
    \newcommand{\CharTok}[1]{\textcolor[rgb]{0.25,0.44,0.63}{{#1}}}
    \newcommand{\StringTok}[1]{\textcolor[rgb]{0.25,0.44,0.63}{{#1}}}
    \newcommand{\CommentTok}[1]{\textcolor[rgb]{0.38,0.63,0.69}{\textit{{#1}}}}
    \newcommand{\OtherTok}[1]{\textcolor[rgb]{0.00,0.44,0.13}{{#1}}}
    \newcommand{\AlertTok}[1]{\textcolor[rgb]{1.00,0.00,0.00}{\textbf{{#1}}}}
    \newcommand{\FunctionTok}[1]{\textcolor[rgb]{0.02,0.16,0.49}{{#1}}}
    \newcommand{\RegionMarkerTok}[1]{{#1}}
    \newcommand{\ErrorTok}[1]{\textcolor[rgb]{1.00,0.00,0.00}{\textbf{{#1}}}}
    \newcommand{\NormalTok}[1]{{#1}}
    
    % Additional commands for more recent versions of Pandoc
    \newcommand{\ConstantTok}[1]{\textcolor[rgb]{0.53,0.00,0.00}{{#1}}}
    \newcommand{\SpecialCharTok}[1]{\textcolor[rgb]{0.25,0.44,0.63}{{#1}}}
    \newcommand{\VerbatimStringTok}[1]{\textcolor[rgb]{0.25,0.44,0.63}{{#1}}}
    \newcommand{\SpecialStringTok}[1]{\textcolor[rgb]{0.73,0.40,0.53}{{#1}}}
    \newcommand{\ImportTok}[1]{{#1}}
    \newcommand{\DocumentationTok}[1]{\textcolor[rgb]{0.73,0.13,0.13}{\textit{{#1}}}}
    \newcommand{\AnnotationTok}[1]{\textcolor[rgb]{0.38,0.63,0.69}{\textbf{\textit{{#1}}}}}
    \newcommand{\CommentVarTok}[1]{\textcolor[rgb]{0.38,0.63,0.69}{\textbf{\textit{{#1}}}}}
    \newcommand{\VariableTok}[1]{\textcolor[rgb]{0.10,0.09,0.49}{{#1}}}
    \newcommand{\ControlFlowTok}[1]{\textcolor[rgb]{0.00,0.44,0.13}{\textbf{{#1}}}}
    \newcommand{\OperatorTok}[1]{\textcolor[rgb]{0.40,0.40,0.40}{{#1}}}
    \newcommand{\BuiltInTok}[1]{{#1}}
    \newcommand{\ExtensionTok}[1]{{#1}}
    \newcommand{\PreprocessorTok}[1]{\textcolor[rgb]{0.74,0.48,0.00}{{#1}}}
    \newcommand{\AttributeTok}[1]{\textcolor[rgb]{0.49,0.56,0.16}{{#1}}}
    \newcommand{\InformationTok}[1]{\textcolor[rgb]{0.38,0.63,0.69}{\textbf{\textit{{#1}}}}}
    \newcommand{\WarningTok}[1]{\textcolor[rgb]{0.38,0.63,0.69}{\textbf{\textit{{#1}}}}}
    
    
    % Define a nice break command that doesn't care if a line doesn't already
    % exist.
    \def\br{\hspace*{\fill} \\* }
    % Math Jax compatibility definitions
    \def\gt{>}
    \def\lt{<}
    \let\Oldtex\TeX
    \let\Oldlatex\LaTeX
    \renewcommand{\TeX}{\textrm{\Oldtex}}
    \renewcommand{\LaTeX}{\textrm{\Oldlatex}}
    % Document parameters
    % Document title
    \title{exploring\_word\_vectors}
    
    
    
    
    

    % Pygments definitions
    
\makeatletter
\def\PY@reset{\let\PY@it=\relax \let\PY@bf=\relax%
    \let\PY@ul=\relax \let\PY@tc=\relax%
    \let\PY@bc=\relax \let\PY@ff=\relax}
\def\PY@tok#1{\csname PY@tok@#1\endcsname}
\def\PY@toks#1+{\ifx\relax#1\empty\else%
    \PY@tok{#1}\expandafter\PY@toks\fi}
\def\PY@do#1{\PY@bc{\PY@tc{\PY@ul{%
    \PY@it{\PY@bf{\PY@ff{#1}}}}}}}
\def\PY#1#2{\PY@reset\PY@toks#1+\relax+\PY@do{#2}}

\expandafter\def\csname PY@tok@w\endcsname{\def\PY@tc##1{\textcolor[rgb]{0.73,0.73,0.73}{##1}}}
\expandafter\def\csname PY@tok@c\endcsname{\let\PY@it=\textit\def\PY@tc##1{\textcolor[rgb]{0.25,0.50,0.50}{##1}}}
\expandafter\def\csname PY@tok@cp\endcsname{\def\PY@tc##1{\textcolor[rgb]{0.74,0.48,0.00}{##1}}}
\expandafter\def\csname PY@tok@k\endcsname{\let\PY@bf=\textbf\def\PY@tc##1{\textcolor[rgb]{0.00,0.50,0.00}{##1}}}
\expandafter\def\csname PY@tok@kp\endcsname{\def\PY@tc##1{\textcolor[rgb]{0.00,0.50,0.00}{##1}}}
\expandafter\def\csname PY@tok@kt\endcsname{\def\PY@tc##1{\textcolor[rgb]{0.69,0.00,0.25}{##1}}}
\expandafter\def\csname PY@tok@o\endcsname{\def\PY@tc##1{\textcolor[rgb]{0.40,0.40,0.40}{##1}}}
\expandafter\def\csname PY@tok@ow\endcsname{\let\PY@bf=\textbf\def\PY@tc##1{\textcolor[rgb]{0.67,0.13,1.00}{##1}}}
\expandafter\def\csname PY@tok@nb\endcsname{\def\PY@tc##1{\textcolor[rgb]{0.00,0.50,0.00}{##1}}}
\expandafter\def\csname PY@tok@nf\endcsname{\def\PY@tc##1{\textcolor[rgb]{0.00,0.00,1.00}{##1}}}
\expandafter\def\csname PY@tok@nc\endcsname{\let\PY@bf=\textbf\def\PY@tc##1{\textcolor[rgb]{0.00,0.00,1.00}{##1}}}
\expandafter\def\csname PY@tok@nn\endcsname{\let\PY@bf=\textbf\def\PY@tc##1{\textcolor[rgb]{0.00,0.00,1.00}{##1}}}
\expandafter\def\csname PY@tok@ne\endcsname{\let\PY@bf=\textbf\def\PY@tc##1{\textcolor[rgb]{0.82,0.25,0.23}{##1}}}
\expandafter\def\csname PY@tok@nv\endcsname{\def\PY@tc##1{\textcolor[rgb]{0.10,0.09,0.49}{##1}}}
\expandafter\def\csname PY@tok@no\endcsname{\def\PY@tc##1{\textcolor[rgb]{0.53,0.00,0.00}{##1}}}
\expandafter\def\csname PY@tok@nl\endcsname{\def\PY@tc##1{\textcolor[rgb]{0.63,0.63,0.00}{##1}}}
\expandafter\def\csname PY@tok@ni\endcsname{\let\PY@bf=\textbf\def\PY@tc##1{\textcolor[rgb]{0.60,0.60,0.60}{##1}}}
\expandafter\def\csname PY@tok@na\endcsname{\def\PY@tc##1{\textcolor[rgb]{0.49,0.56,0.16}{##1}}}
\expandafter\def\csname PY@tok@nt\endcsname{\let\PY@bf=\textbf\def\PY@tc##1{\textcolor[rgb]{0.00,0.50,0.00}{##1}}}
\expandafter\def\csname PY@tok@nd\endcsname{\def\PY@tc##1{\textcolor[rgb]{0.67,0.13,1.00}{##1}}}
\expandafter\def\csname PY@tok@s\endcsname{\def\PY@tc##1{\textcolor[rgb]{0.73,0.13,0.13}{##1}}}
\expandafter\def\csname PY@tok@sd\endcsname{\let\PY@it=\textit\def\PY@tc##1{\textcolor[rgb]{0.73,0.13,0.13}{##1}}}
\expandafter\def\csname PY@tok@si\endcsname{\let\PY@bf=\textbf\def\PY@tc##1{\textcolor[rgb]{0.73,0.40,0.53}{##1}}}
\expandafter\def\csname PY@tok@se\endcsname{\let\PY@bf=\textbf\def\PY@tc##1{\textcolor[rgb]{0.73,0.40,0.13}{##1}}}
\expandafter\def\csname PY@tok@sr\endcsname{\def\PY@tc##1{\textcolor[rgb]{0.73,0.40,0.53}{##1}}}
\expandafter\def\csname PY@tok@ss\endcsname{\def\PY@tc##1{\textcolor[rgb]{0.10,0.09,0.49}{##1}}}
\expandafter\def\csname PY@tok@sx\endcsname{\def\PY@tc##1{\textcolor[rgb]{0.00,0.50,0.00}{##1}}}
\expandafter\def\csname PY@tok@m\endcsname{\def\PY@tc##1{\textcolor[rgb]{0.40,0.40,0.40}{##1}}}
\expandafter\def\csname PY@tok@gh\endcsname{\let\PY@bf=\textbf\def\PY@tc##1{\textcolor[rgb]{0.00,0.00,0.50}{##1}}}
\expandafter\def\csname PY@tok@gu\endcsname{\let\PY@bf=\textbf\def\PY@tc##1{\textcolor[rgb]{0.50,0.00,0.50}{##1}}}
\expandafter\def\csname PY@tok@gd\endcsname{\def\PY@tc##1{\textcolor[rgb]{0.63,0.00,0.00}{##1}}}
\expandafter\def\csname PY@tok@gi\endcsname{\def\PY@tc##1{\textcolor[rgb]{0.00,0.63,0.00}{##1}}}
\expandafter\def\csname PY@tok@gr\endcsname{\def\PY@tc##1{\textcolor[rgb]{1.00,0.00,0.00}{##1}}}
\expandafter\def\csname PY@tok@ge\endcsname{\let\PY@it=\textit}
\expandafter\def\csname PY@tok@gs\endcsname{\let\PY@bf=\textbf}
\expandafter\def\csname PY@tok@gp\endcsname{\let\PY@bf=\textbf\def\PY@tc##1{\textcolor[rgb]{0.00,0.00,0.50}{##1}}}
\expandafter\def\csname PY@tok@go\endcsname{\def\PY@tc##1{\textcolor[rgb]{0.53,0.53,0.53}{##1}}}
\expandafter\def\csname PY@tok@gt\endcsname{\def\PY@tc##1{\textcolor[rgb]{0.00,0.27,0.87}{##1}}}
\expandafter\def\csname PY@tok@err\endcsname{\def\PY@bc##1{\setlength{\fboxsep}{0pt}\fcolorbox[rgb]{1.00,0.00,0.00}{1,1,1}{\strut ##1}}}
\expandafter\def\csname PY@tok@kc\endcsname{\let\PY@bf=\textbf\def\PY@tc##1{\textcolor[rgb]{0.00,0.50,0.00}{##1}}}
\expandafter\def\csname PY@tok@kd\endcsname{\let\PY@bf=\textbf\def\PY@tc##1{\textcolor[rgb]{0.00,0.50,0.00}{##1}}}
\expandafter\def\csname PY@tok@kn\endcsname{\let\PY@bf=\textbf\def\PY@tc##1{\textcolor[rgb]{0.00,0.50,0.00}{##1}}}
\expandafter\def\csname PY@tok@kr\endcsname{\let\PY@bf=\textbf\def\PY@tc##1{\textcolor[rgb]{0.00,0.50,0.00}{##1}}}
\expandafter\def\csname PY@tok@bp\endcsname{\def\PY@tc##1{\textcolor[rgb]{0.00,0.50,0.00}{##1}}}
\expandafter\def\csname PY@tok@fm\endcsname{\def\PY@tc##1{\textcolor[rgb]{0.00,0.00,1.00}{##1}}}
\expandafter\def\csname PY@tok@vc\endcsname{\def\PY@tc##1{\textcolor[rgb]{0.10,0.09,0.49}{##1}}}
\expandafter\def\csname PY@tok@vg\endcsname{\def\PY@tc##1{\textcolor[rgb]{0.10,0.09,0.49}{##1}}}
\expandafter\def\csname PY@tok@vi\endcsname{\def\PY@tc##1{\textcolor[rgb]{0.10,0.09,0.49}{##1}}}
\expandafter\def\csname PY@tok@vm\endcsname{\def\PY@tc##1{\textcolor[rgb]{0.10,0.09,0.49}{##1}}}
\expandafter\def\csname PY@tok@sa\endcsname{\def\PY@tc##1{\textcolor[rgb]{0.73,0.13,0.13}{##1}}}
\expandafter\def\csname PY@tok@sb\endcsname{\def\PY@tc##1{\textcolor[rgb]{0.73,0.13,0.13}{##1}}}
\expandafter\def\csname PY@tok@sc\endcsname{\def\PY@tc##1{\textcolor[rgb]{0.73,0.13,0.13}{##1}}}
\expandafter\def\csname PY@tok@dl\endcsname{\def\PY@tc##1{\textcolor[rgb]{0.73,0.13,0.13}{##1}}}
\expandafter\def\csname PY@tok@s2\endcsname{\def\PY@tc##1{\textcolor[rgb]{0.73,0.13,0.13}{##1}}}
\expandafter\def\csname PY@tok@sh\endcsname{\def\PY@tc##1{\textcolor[rgb]{0.73,0.13,0.13}{##1}}}
\expandafter\def\csname PY@tok@s1\endcsname{\def\PY@tc##1{\textcolor[rgb]{0.73,0.13,0.13}{##1}}}
\expandafter\def\csname PY@tok@mb\endcsname{\def\PY@tc##1{\textcolor[rgb]{0.40,0.40,0.40}{##1}}}
\expandafter\def\csname PY@tok@mf\endcsname{\def\PY@tc##1{\textcolor[rgb]{0.40,0.40,0.40}{##1}}}
\expandafter\def\csname PY@tok@mh\endcsname{\def\PY@tc##1{\textcolor[rgb]{0.40,0.40,0.40}{##1}}}
\expandafter\def\csname PY@tok@mi\endcsname{\def\PY@tc##1{\textcolor[rgb]{0.40,0.40,0.40}{##1}}}
\expandafter\def\csname PY@tok@il\endcsname{\def\PY@tc##1{\textcolor[rgb]{0.40,0.40,0.40}{##1}}}
\expandafter\def\csname PY@tok@mo\endcsname{\def\PY@tc##1{\textcolor[rgb]{0.40,0.40,0.40}{##1}}}
\expandafter\def\csname PY@tok@ch\endcsname{\let\PY@it=\textit\def\PY@tc##1{\textcolor[rgb]{0.25,0.50,0.50}{##1}}}
\expandafter\def\csname PY@tok@cm\endcsname{\let\PY@it=\textit\def\PY@tc##1{\textcolor[rgb]{0.25,0.50,0.50}{##1}}}
\expandafter\def\csname PY@tok@cpf\endcsname{\let\PY@it=\textit\def\PY@tc##1{\textcolor[rgb]{0.25,0.50,0.50}{##1}}}
\expandafter\def\csname PY@tok@c1\endcsname{\let\PY@it=\textit\def\PY@tc##1{\textcolor[rgb]{0.25,0.50,0.50}{##1}}}
\expandafter\def\csname PY@tok@cs\endcsname{\let\PY@it=\textit\def\PY@tc##1{\textcolor[rgb]{0.25,0.50,0.50}{##1}}}

\def\PYZbs{\char`\\}
\def\PYZus{\char`\_}
\def\PYZob{\char`\{}
\def\PYZcb{\char`\}}
\def\PYZca{\char`\^}
\def\PYZam{\char`\&}
\def\PYZlt{\char`\<}
\def\PYZgt{\char`\>}
\def\PYZsh{\char`\#}
\def\PYZpc{\char`\%}
\def\PYZdl{\char`\$}
\def\PYZhy{\char`\-}
\def\PYZsq{\char`\'}
\def\PYZdq{\char`\"}
\def\PYZti{\char`\~}
% for compatibility with earlier versions
\def\PYZat{@}
\def\PYZlb{[}
\def\PYZrb{]}
\makeatother


    % Exact colors from NB
    \definecolor{incolor}{rgb}{0.0, 0.0, 0.5}
    \definecolor{outcolor}{rgb}{0.545, 0.0, 0.0}



    
    % Prevent overflowing lines due to hard-to-break entities
    \sloppy 
    % Setup hyperref package
    \hypersetup{
      breaklinks=true,  % so long urls are correctly broken across lines
      colorlinks=true,
      urlcolor=urlcolor,
      linkcolor=linkcolor,
      citecolor=citecolor,
      }
    % Slightly bigger margins than the latex defaults
    
    \geometry{verbose,tmargin=1in,bmargin=1in,lmargin=1in,rmargin=1in}
    
    

    \begin{document}
    
    
    \maketitle
    
    

    
    \hypertarget{cs224n-assignment-1-exploring-word-vectors-25-points}{%
\section{CS224N Assignment 1: Exploring Word Vectors (25
Points)}\label{cs224n-assignment-1-exploring-word-vectors-25-points}}

\hypertarget{due-430pm-tue-jan-14}{%
\subsubsection{ Due 4:30pm, Tue Jan 14 }\label{due-430pm-tue-jan-14}}

Welcome to CS224n!

Before you start, make sure you read the README.txt in the same
directory as this notebook. You will find many provided codes in the
notebook. We highly encourage you to read and understand the provided
codes as part of the learning :-)

    \begin{Verbatim}[commandchars=\\\{\}]
{\color{incolor}In [{\color{incolor}1}]:} \PY{c+c1}{\PYZsh{} All Import Statements Defined Here}
        \PY{c+c1}{\PYZsh{} Note: Do not add to this list.}
        \PY{c+c1}{\PYZsh{} \PYZhy{}\PYZhy{}\PYZhy{}\PYZhy{}\PYZhy{}\PYZhy{}\PYZhy{}\PYZhy{}\PYZhy{}\PYZhy{}\PYZhy{}\PYZhy{}\PYZhy{}\PYZhy{}\PYZhy{}\PYZhy{}}
        
        \PY{k+kn}{import} \PY{n+nn}{sys}
        \PY{k}{assert} \PY{n}{sys}\PY{o}{.}\PY{n}{version\PYZus{}info}\PY{p}{[}\PY{l+m+mi}{0}\PY{p}{]}\PY{o}{==}\PY{l+m+mi}{3}
        \PY{k}{assert} \PY{n}{sys}\PY{o}{.}\PY{n}{version\PYZus{}info}\PY{p}{[}\PY{l+m+mi}{1}\PY{p}{]} \PY{o}{\PYZgt{}}\PY{o}{=} \PY{l+m+mi}{5}
        
        \PY{k+kn}{from} \PY{n+nn}{gensim}\PY{n+nn}{.}\PY{n+nn}{models} \PY{k}{import} \PY{n}{KeyedVectors}
        \PY{k+kn}{from} \PY{n+nn}{gensim}\PY{n+nn}{.}\PY{n+nn}{test}\PY{n+nn}{.}\PY{n+nn}{utils} \PY{k}{import} \PY{n}{datapath}
        \PY{k+kn}{import} \PY{n+nn}{pprint}
        \PY{k+kn}{import} \PY{n+nn}{matplotlib}\PY{n+nn}{.}\PY{n+nn}{pyplot} \PY{k}{as} \PY{n+nn}{plt}
        \PY{n}{plt}\PY{o}{.}\PY{n}{rcParams}\PY{p}{[}\PY{l+s+s1}{\PYZsq{}}\PY{l+s+s1}{figure.figsize}\PY{l+s+s1}{\PYZsq{}}\PY{p}{]} \PY{o}{=} \PY{p}{[}\PY{l+m+mi}{10}\PY{p}{,} \PY{l+m+mi}{5}\PY{p}{]}
        \PY{k+kn}{import} \PY{n+nn}{nltk}
        \PY{n}{nltk}\PY{o}{.}\PY{n}{download}\PY{p}{(}\PY{l+s+s1}{\PYZsq{}}\PY{l+s+s1}{reuters}\PY{l+s+s1}{\PYZsq{}}\PY{p}{)}
        \PY{k+kn}{from} \PY{n+nn}{nltk}\PY{n+nn}{.}\PY{n+nn}{corpus} \PY{k}{import} \PY{n}{reuters}
        \PY{k+kn}{import} \PY{n+nn}{numpy} \PY{k}{as} \PY{n+nn}{np}
        \PY{k+kn}{import} \PY{n+nn}{random}
        \PY{k+kn}{import} \PY{n+nn}{scipy} \PY{k}{as} \PY{n+nn}{sp}
        \PY{k+kn}{from} \PY{n+nn}{sklearn}\PY{n+nn}{.}\PY{n+nn}{decomposition} \PY{k}{import} \PY{n}{TruncatedSVD}
        \PY{k+kn}{from} \PY{n+nn}{sklearn}\PY{n+nn}{.}\PY{n+nn}{decomposition} \PY{k}{import} \PY{n}{PCA}
        
        \PY{n}{START\PYZus{}TOKEN} \PY{o}{=} \PY{l+s+s1}{\PYZsq{}}\PY{l+s+s1}{\PYZlt{}START\PYZgt{}}\PY{l+s+s1}{\PYZsq{}}
        \PY{n}{END\PYZus{}TOKEN} \PY{o}{=} \PY{l+s+s1}{\PYZsq{}}\PY{l+s+s1}{\PYZlt{}END\PYZgt{}}\PY{l+s+s1}{\PYZsq{}}
        
        \PY{n}{np}\PY{o}{.}\PY{n}{random}\PY{o}{.}\PY{n}{seed}\PY{p}{(}\PY{l+m+mi}{0}\PY{p}{)}
        \PY{n}{random}\PY{o}{.}\PY{n}{seed}\PY{p}{(}\PY{l+m+mi}{0}\PY{p}{)}
        \PY{c+c1}{\PYZsh{} \PYZhy{}\PYZhy{}\PYZhy{}\PYZhy{}\PYZhy{}\PYZhy{}\PYZhy{}\PYZhy{}\PYZhy{}\PYZhy{}\PYZhy{}\PYZhy{}\PYZhy{}\PYZhy{}\PYZhy{}\PYZhy{}}
\end{Verbatim}

    \begin{Verbatim}[commandchars=\\\{\}]
[nltk\_data] Downloading package reuters to
[nltk\_data]     C:\textbackslash{}Users\textbackslash{}orion.darley\textbackslash{}AppData\textbackslash{}Roaming\textbackslash{}nltk\_data{\ldots}
[nltk\_data]   Package reuters is already up-to-date!

    \end{Verbatim}

    \hypertarget{word-vectors}{%
\subsection{Word Vectors}\label{word-vectors}}

Word Vectors are often used as a fundamental component for downstream
NLP tasks, e.g.~question answering, text generation, translation, etc.,
so it is important to build some intuitions as to their strengths and
weaknesses. Here, you will explore two types of word vectors: those
derived from \emph{co-occurrence matrices}, and those derived via
\emph{GloVe}.

\textbf{Assignment Notes:} Please make sure to save the notebook as you
go along. Submission Instructions are located at the bottom of the
notebook.

\textbf{Note on Terminology:} The terms ``word vectors'' and ``word
embeddings'' are often used interchangeably. The term ``embedding''
refers to the fact that we are encoding aspects of a word's meaning in a
lower dimensional space. As
\href{https://en.wikipedia.org/wiki/Word_embedding}{Wikipedia} states,
``\emph{conceptually it involves a mathematical embedding from a space
with one dimension per word to a continuous vector space with a much
lower dimension}''.

    \hypertarget{part-1-count-based-word-vectors-10-points}{%
\subsection{Part 1: Count-Based Word Vectors (10
points)}\label{part-1-count-based-word-vectors-10-points}}

Most word vector models start from the following idea:

\emph{You shall know a word by the company it keeps
(\href{https://en.wikipedia.org/wiki/John_Rupert_Firth}{Firth, J. R.
1957:11})}

Many word vector implementations are driven by the idea that similar
words, i.e., (near) synonyms, will be used in similar contexts. As a
result, similar words will often be spoken or written along with a
shared subset of words, i.e., contexts. By examining these contexts, we
can try to develop embeddings for our words. With this intuition in
mind, many ``old school'' approaches to constructing word vectors relied
on word counts. Here we elaborate upon one of those strategies,
\emph{co-occurrence matrices} (for more information, see
\href{http://web.stanford.edu/class/cs124/lec/vectorsemantics.video.pdf}{here}
or
\href{https://medium.com/data-science-group-iitr/word-embedding-2d05d270b285}{here}).

    \hypertarget{co-occurrence}{%
\subsubsection{Co-Occurrence}\label{co-occurrence}}

A co-occurrence matrix counts how often things co-occur in some
environment. Given some word \(w_i\) occurring in the document, we
consider the \emph{context window} surrounding \(w_i\). Supposing our
fixed window size is \(n\), then this is the \(n\) preceding and \(n\)
subsequent words in that document, i.e.~words \(w_{i-n} \dots w_{i-1}\)
and \(w_{i+1} \dots w_{i+n}\). We build a \emph{co-occurrence matrix}
\(M\), which is a symmetric word-by-word matrix in which \(M_{ij}\) is
the number of times \(w_j\) appears inside \(w_i\)'s window among all
documents.

\textbf{Example: Co-Occurrence with Fixed Window of n=1}:

Document 1: ``all that glitters is not gold''

Document 2: ``all is well that ends well''

\begin{longtable}[]{@{}lllllllllll@{}}
\toprule
* & \texttt{\textless{}START\textgreater{}} & all & that & glitters & is
& not & gold & well & ends &
\texttt{\textless{}END\textgreater{}}\tabularnewline
\midrule
\endhead
\texttt{\textless{}START\textgreater{}} & 0 & 2 & 0 & 0 & 0 & 0 & 0 & 0
& 0 & 0\tabularnewline
all & 2 & 0 & 1 & 0 & 1 & 0 & 0 & 0 & 0 & 0\tabularnewline
that & 0 & 1 & 0 & 1 & 0 & 0 & 0 & 1 & 1 & 0\tabularnewline
glitters & 0 & 0 & 1 & 0 & 1 & 0 & 0 & 0 & 0 & 0\tabularnewline
is & 0 & 1 & 0 & 1 & 0 & 1 & 0 & 1 & 0 & 0\tabularnewline
not & 0 & 0 & 0 & 0 & 1 & 0 & 1 & 0 & 0 & 0\tabularnewline
gold & 0 & 0 & 0 & 0 & 0 & 1 & 0 & 0 & 0 & 1\tabularnewline
well & 0 & 0 & 1 & 0 & 1 & 0 & 0 & 0 & 1 & 1\tabularnewline
ends & 0 & 0 & 1 & 0 & 0 & 0 & 0 & 1 & 0 & 0\tabularnewline
\texttt{\textless{}END\textgreater{}} & 0 & 0 & 0 & 0 & 0 & 0 & 1 & 1 &
0 & 0\tabularnewline
\bottomrule
\end{longtable}

\textbf{Note:} In NLP, we often add
\texttt{\textless{}START\textgreater{}} and
\texttt{\textless{}END\textgreater{}} tokens to represent the beginning
and end of sentences, paragraphs or documents. In thise case we imagine
\texttt{\textless{}START\textgreater{}} and
\texttt{\textless{}END\textgreater{}} tokens encapsulating each
document, e.g., ``\texttt{\textless{}START\textgreater{}} All that
glitters is not gold \texttt{\textless{}END\textgreater{}}'', and
include these tokens in our co-occurrence counts.

The rows (or columns) of this matrix provide one type of word vectors
(those based on word-word co-occurrence), but the vectors will be large
in general (linear in the number of distinct words in a corpus). Thus,
our next step is to run \emph{dimensionality reduction}. In particular,
we will run \emph{SVD (Singular Value Decomposition)}, which is a kind
of generalized \emph{PCA (Principal Components Analysis)} to select the
top \(k\) principal components. Here's a visualization of dimensionality
reduction with SVD. In this picture our co-occurrence matrix is \(A\)
with \(n\) rows corresponding to \(n\) words. We obtain a full matrix
decomposition, with the singular values ordered in the diagonal \(S\)
matrix, and our new, shorter length-\(k\) word vectors in \(U_k\).

\begin{figure}
\centering
\includegraphics{./imgs/svd.png}
\caption{Picture of an SVD}
\end{figure}

This reduced-dimensionality co-occurrence representation preserves
semantic relationships between words, e.g. \emph{doctor} and
\emph{hospital} will be closer than \emph{doctor} and \emph{dog}.

\textbf{Notes:} If you can barely remember what an eigenvalue is, here's
\href{https://davetang.org/file/Singular_Value_Decomposition_Tutorial.pdf}{a
slow, friendly introduction to SVD}. If you want to learn more
thoroughly about PCA or SVD, feel free to check out lectures
\href{https://web.stanford.edu/class/cs168/l/l7.pdf}{7},
\href{http://theory.stanford.edu/~tim/s15/l/l8.pdf}{8}, and
\href{https://web.stanford.edu/class/cs168/l/l9.pdf}{9} of CS168. These
course notes provide a great high-level treatment of these general
purpose algorithms. Though, for the purpose of this class, you only need
to know how to extract the k-dimensional embeddings by utilizing
pre-programmed implementations of these algorithms from the numpy,
scipy, or sklearn python packages. In practice, it is challenging to
apply full SVD to large corpora because of the memory needed to perform
PCA or SVD. However, if you only want the top \(k\) vector components
for relatively small \(k\) --- known as
\href{https://en.wikipedia.org/wiki/Singular_value_decomposition\#Truncated_SVD}{Truncated
SVD} --- then there are reasonably scalable techniques to compute those
iteratively.

    \hypertarget{plotting-co-occurrence-word-embeddings}{%
\subsubsection{Plotting Co-Occurrence Word
Embeddings}\label{plotting-co-occurrence-word-embeddings}}

Here, we will be using the Reuters (business and financial news) corpus.
If you haven't run the import cell at the top of this page, please run
it now (click it and press SHIFT-RETURN). The corpus consists of 10,788
news documents totaling 1.3 million words. These documents span 90
categories and are split into train and test. For more details, please
see https://www.nltk.org/book/ch02.html. We provide a
\texttt{read\_corpus} function below that pulls out only articles from
the ``crude'' (i.e.~news articles about oil, gas, etc.) category. The
function also adds \texttt{\textless{}START\textgreater{}} and
\texttt{\textless{}END\textgreater{}} tokens to each of the documents,
and lowercases words. You do \textbf{not} have to perform any other kind
of pre-processing.

    \begin{Verbatim}[commandchars=\\\{\}]
{\color{incolor}In [{\color{incolor}2}]:} \PY{k}{def} \PY{n+nf}{read\PYZus{}corpus}\PY{p}{(}\PY{n}{category}\PY{o}{=}\PY{l+s+s2}{\PYZdq{}}\PY{l+s+s2}{crude}\PY{l+s+s2}{\PYZdq{}}\PY{p}{)}\PY{p}{:}
            \PY{l+s+sd}{\PYZdq{}\PYZdq{}\PYZdq{} Read files from the specified Reuter\PYZsq{}s category.}
        \PY{l+s+sd}{        Params:}
        \PY{l+s+sd}{            category (string): category name}
        \PY{l+s+sd}{        Return:}
        \PY{l+s+sd}{            list of lists, with words from each of the processed files}
        \PY{l+s+sd}{    \PYZdq{}\PYZdq{}\PYZdq{}}
            \PY{n}{files} \PY{o}{=} \PY{n}{reuters}\PY{o}{.}\PY{n}{fileids}\PY{p}{(}\PY{n}{category}\PY{p}{)}
            \PY{k}{return} \PY{p}{[}\PY{p}{[}\PY{n}{START\PYZus{}TOKEN}\PY{p}{]} \PY{o}{+} \PY{p}{[}\PY{n}{w}\PY{o}{.}\PY{n}{lower}\PY{p}{(}\PY{p}{)} \PY{k}{for} \PY{n}{w} \PY{o+ow}{in} \PY{n+nb}{list}\PY{p}{(}\PY{n}{reuters}\PY{o}{.}\PY{n}{words}\PY{p}{(}\PY{n}{f}\PY{p}{)}\PY{p}{)}\PY{p}{]} \PY{o}{+} \PY{p}{[}\PY{n}{END\PYZus{}TOKEN}\PY{p}{]} \PY{k}{for} \PY{n}{f} \PY{o+ow}{in} \PY{n}{files}\PY{p}{]}
\end{Verbatim}

    Let's have a look what these documents are like\ldots{}.

    \begin{Verbatim}[commandchars=\\\{\}]
{\color{incolor}In [{\color{incolor}3}]:} \PY{n}{reuters\PYZus{}corpus} \PY{o}{=} \PY{n}{read\PYZus{}corpus}\PY{p}{(}\PY{p}{)}
        \PY{n}{pprint}\PY{o}{.}\PY{n}{pprint}\PY{p}{(}\PY{n}{reuters\PYZus{}corpus}\PY{p}{[}\PY{p}{:}\PY{l+m+mi}{3}\PY{p}{]}\PY{p}{,} \PY{n}{compact}\PY{o}{=}\PY{k+kc}{True}\PY{p}{,} \PY{n}{width}\PY{o}{=}\PY{l+m+mi}{100}\PY{p}{)}
\end{Verbatim}

    \begin{Verbatim}[commandchars=\\\{\}]
[['<START>', 'japan', 'to', 'revise', 'long', '-', 'term', 'energy', 'demand', 'downwards', 'the',
  'ministry', 'of', 'international', 'trade', 'and', 'industry', '(', 'miti', ')', 'will', 'revise',
  'its', 'long', '-', 'term', 'energy', 'supply', '/', 'demand', 'outlook', 'by', 'august', 'to',
  'meet', 'a', 'forecast', 'downtrend', 'in', 'japanese', 'energy', 'demand', ',', 'ministry',
  'officials', 'said', '.', 'miti', 'is', 'expected', 'to', 'lower', 'the', 'projection', 'for',
  'primary', 'energy', 'supplies', 'in', 'the', 'year', '2000', 'to', '550', 'mln', 'kilolitres',
  '(', 'kl', ')', 'from', '600', 'mln', ',', 'they', 'said', '.', 'the', 'decision', 'follows',
  'the', 'emergence', 'of', 'structural', 'changes', 'in', 'japanese', 'industry', 'following',
  'the', 'rise', 'in', 'the', 'value', 'of', 'the', 'yen', 'and', 'a', 'decline', 'in', 'domestic',
  'electric', 'power', 'demand', '.', 'miti', 'is', 'planning', 'to', 'work', 'out', 'a', 'revised',
  'energy', 'supply', '/', 'demand', 'outlook', 'through', 'deliberations', 'of', 'committee',
  'meetings', 'of', 'the', 'agency', 'of', 'natural', 'resources', 'and', 'energy', ',', 'the',
  'officials', 'said', '.', 'they', 'said', 'miti', 'will', 'also', 'review', 'the', 'breakdown',
  'of', 'energy', 'supply', 'sources', ',', 'including', 'oil', ',', 'nuclear', ',', 'coal', 'and',
  'natural', 'gas', '.', 'nuclear', 'energy', 'provided', 'the', 'bulk', 'of', 'japan', "'", 's',
  'electric', 'power', 'in', 'the', 'fiscal', 'year', 'ended', 'march', '31', ',', 'supplying',
  'an', 'estimated', '27', 'pct', 'on', 'a', 'kilowatt', '/', 'hour', 'basis', ',', 'followed',
  'by', 'oil', '(', '23', 'pct', ')', 'and', 'liquefied', 'natural', 'gas', '(', '21', 'pct', '),',
  'they', 'noted', '.', '<END>'],
 ['<START>', 'energy', '/', 'u', '.', 's', '.', 'petrochemical', 'industry', 'cheap', 'oil',
  'feedstocks', ',', 'the', 'weakened', 'u', '.', 's', '.', 'dollar', 'and', 'a', 'plant',
  'utilization', 'rate', 'approaching', '90', 'pct', 'will', 'propel', 'the', 'streamlined', 'u',
  '.', 's', '.', 'petrochemical', 'industry', 'to', 'record', 'profits', 'this', 'year', ',',
  'with', 'growth', 'expected', 'through', 'at', 'least', '1990', ',', 'major', 'company',
  'executives', 'predicted', '.', 'this', 'bullish', 'outlook', 'for', 'chemical', 'manufacturing',
  'and', 'an', 'industrywide', 'move', 'to', 'shed', 'unrelated', 'businesses', 'has', 'prompted',
  'gaf', 'corp', '\&', 'lt', ';', 'gaf', '>,', 'privately', '-', 'held', 'cain', 'chemical', 'inc',
  ',', 'and', 'other', 'firms', 'to', 'aggressively', 'seek', 'acquisitions', 'of', 'petrochemical',
  'plants', '.', 'oil', 'companies', 'such', 'as', 'ashland', 'oil', 'inc', '\&', 'lt', ';', 'ash',
  '>,', 'the', 'kentucky', '-', 'based', 'oil', 'refiner', 'and', 'marketer', ',', 'are', 'also',
  'shopping', 'for', 'money', '-', 'making', 'petrochemical', 'businesses', 'to', 'buy', '.', '"',
  'i', 'see', 'us', 'poised', 'at', 'the', 'threshold', 'of', 'a', 'golden', 'period', ',"', 'said',
  'paul', 'oreffice', ',', 'chairman', 'of', 'giant', 'dow', 'chemical', 'co', '\&', 'lt', ';',
  'dow', '>,', 'adding', ',', '"', 'there', "'", 's', 'no', 'major', 'plant', 'capacity', 'being',
  'added', 'around', 'the', 'world', 'now', '.', 'the', 'whole', 'game', 'is', 'bringing', 'out',
  'new', 'products', 'and', 'improving', 'the', 'old', 'ones', '."', 'analysts', 'say', 'the',
  'chemical', 'industry', "'", 's', 'biggest', 'customers', ',', 'automobile', 'manufacturers',
  'and', 'home', 'builders', 'that', 'use', 'a', 'lot', 'of', 'paints', 'and', 'plastics', ',',
  'are', 'expected', 'to', 'buy', 'quantities', 'this', 'year', '.', 'u', '.', 's', '.',
  'petrochemical', 'plants', 'are', 'currently', 'operating', 'at', 'about', '90', 'pct',
  'capacity', ',', 'reflecting', 'tighter', 'supply', 'that', 'could', 'hike', 'product', 'prices',
  'by', '30', 'to', '40', 'pct', 'this', 'year', ',', 'said', 'john', 'dosher', ',', 'managing',
  'director', 'of', 'pace', 'consultants', 'inc', 'of', 'houston', '.', 'demand', 'for', 'some',
  'products', 'such', 'as', 'styrene', 'could', 'push', 'profit', 'margins', 'up', 'by', 'as',
  'much', 'as', '300', 'pct', ',', 'he', 'said', '.', 'oreffice', ',', 'speaking', 'at', 'a',
  'meeting', 'of', 'chemical', 'engineers', 'in', 'houston', ',', 'said', 'dow', 'would', 'easily',
  'top', 'the', '741', 'mln', 'dlrs', 'it', 'earned', 'last', 'year', 'and', 'predicted', 'it',
  'would', 'have', 'the', 'best', 'year', 'in', 'its', 'history', '.', 'in', '1985', ',', 'when',
  'oil', 'prices', 'were', 'still', 'above', '25', 'dlrs', 'a', 'barrel', 'and', 'chemical',
  'exports', 'were', 'adversely', 'affected', 'by', 'the', 'strong', 'u', '.', 's', '.', 'dollar',
  ',', 'dow', 'had', 'profits', 'of', '58', 'mln', 'dlrs', '.', '"', 'i', 'believe', 'the',
  'entire', 'chemical', 'industry', 'is', 'headed', 'for', 'a', 'record', 'year', 'or', 'close',
  'to', 'it', ',"', 'oreffice', 'said', '.', 'gaf', 'chairman', 'samuel', 'heyman', 'estimated',
  'that', 'the', 'u', '.', 's', '.', 'chemical', 'industry', 'would', 'report', 'a', '20', 'pct',
  'gain', 'in', 'profits', 'during', '1987', '.', 'last', 'year', ',', 'the', 'domestic',
  'industry', 'earned', 'a', 'total', 'of', '13', 'billion', 'dlrs', ',', 'a', '54', 'pct', 'leap',
  'from', '1985', '.', 'the', 'turn', 'in', 'the', 'fortunes', 'of', 'the', 'once', '-', 'sickly',
  'chemical', 'industry', 'has', 'been', 'brought', 'about', 'by', 'a', 'combination', 'of', 'luck',
  'and', 'planning', ',', 'said', 'pace', "'", 's', 'john', 'dosher', '.', 'dosher', 'said', 'last',
  'year', "'", 's', 'fall', 'in', 'oil', 'prices', 'made', 'feedstocks', 'dramatically', 'cheaper',
  'and', 'at', 'the', 'same', 'time', 'the', 'american', 'dollar', 'was', 'weakening', 'against',
  'foreign', 'currencies', '.', 'that', 'helped', 'boost', 'u', '.', 's', '.', 'chemical',
  'exports', '.', 'also', 'helping', 'to', 'bring', 'supply', 'and', 'demand', 'into', 'balance',
  'has', 'been', 'the', 'gradual', 'market', 'absorption', 'of', 'the', 'extra', 'chemical',
  'manufacturing', 'capacity', 'created', 'by', 'middle', 'eastern', 'oil', 'producers', 'in',
  'the', 'early', '1980s', '.', 'finally', ',', 'virtually', 'all', 'major', 'u', '.', 's', '.',
  'chemical', 'manufacturers', 'have', 'embarked', 'on', 'an', 'extensive', 'corporate',
  'restructuring', 'program', 'to', 'mothball', 'inefficient', 'plants', ',', 'trim', 'the',
  'payroll', 'and', 'eliminate', 'unrelated', 'businesses', '.', 'the', 'restructuring', 'touched',
  'off', 'a', 'flurry', 'of', 'friendly', 'and', 'hostile', 'takeover', 'attempts', '.', 'gaf', ',',
  'which', 'made', 'an', 'unsuccessful', 'attempt', 'in', '1985', 'to', 'acquire', 'union',
  'carbide', 'corp', '\&', 'lt', ';', 'uk', '>,', 'recently', 'offered', 'three', 'billion', 'dlrs',
  'for', 'borg', 'warner', 'corp', '\&', 'lt', ';', 'bor', '>,', 'a', 'chicago', 'manufacturer',
  'of', 'plastics', 'and', 'chemicals', '.', 'another', 'industry', 'powerhouse', ',', 'w', '.',
  'r', '.', 'grace', '\&', 'lt', ';', 'gra', '>', 'has', 'divested', 'its', 'retailing', ',',
  'restaurant', 'and', 'fertilizer', 'businesses', 'to', 'raise', 'cash', 'for', 'chemical',
  'acquisitions', '.', 'but', 'some', 'experts', 'worry', 'that', 'the', 'chemical', 'industry',
  'may', 'be', 'headed', 'for', 'trouble', 'if', 'companies', 'continue', 'turning', 'their',
  'back', 'on', 'the', 'manufacturing', 'of', 'staple', 'petrochemical', 'commodities', ',', 'such',
  'as', 'ethylene', ',', 'in', 'favor', 'of', 'more', 'profitable', 'specialty', 'chemicals',
  'that', 'are', 'custom', '-', 'designed', 'for', 'a', 'small', 'group', 'of', 'buyers', '.', '"',
  'companies', 'like', 'dupont', '\&', 'lt', ';', 'dd', '>', 'and', 'monsanto', 'co', '\&', 'lt', ';',
  'mtc', '>', 'spent', 'the', 'past', 'two', 'or', 'three', 'years', 'trying', 'to', 'get', 'out',
  'of', 'the', 'commodity', 'chemical', 'business', 'in', 'reaction', 'to', 'how', 'badly', 'the',
  'market', 'had', 'deteriorated', ',"', 'dosher', 'said', '.', '"', 'but', 'i', 'think', 'they',
  'will', 'eventually', 'kill', 'the', 'margins', 'on', 'the', 'profitable', 'chemicals', 'in',
  'the', 'niche', 'market', '."', 'some', 'top', 'chemical', 'executives', 'share', 'the',
  'concern', '.', '"', 'the', 'challenge', 'for', 'our', 'industry', 'is', 'to', 'keep', 'from',
  'getting', 'carried', 'away', 'and', 'repeating', 'past', 'mistakes', ',"', 'gaf', "'", 's',
  'heyman', 'cautioned', '.', '"', 'the', 'shift', 'from', 'commodity', 'chemicals', 'may', 'be',
  'ill', '-', 'advised', '.', 'specialty', 'businesses', 'do', 'not', 'stay', 'special', 'long',
  '."', 'houston', '-', 'based', 'cain', 'chemical', ',', 'created', 'this', 'month', 'by', 'the',
  'sterling', 'investment', 'banking', 'group', ',', 'believes', 'it', 'can', 'generate', '700',
  'mln', 'dlrs', 'in', 'annual', 'sales', 'by', 'bucking', 'the', 'industry', 'trend', '.',
  'chairman', 'gordon', 'cain', ',', 'who', 'previously', 'led', 'a', 'leveraged', 'buyout', 'of',
  'dupont', "'", 's', 'conoco', 'inc', "'", 's', 'chemical', 'business', ',', 'has', 'spent', '1',
  '.', '1', 'billion', 'dlrs', 'since', 'january', 'to', 'buy', 'seven', 'petrochemical', 'plants',
  'along', 'the', 'texas', 'gulf', 'coast', '.', 'the', 'plants', 'produce', 'only', 'basic',
  'commodity', 'petrochemicals', 'that', 'are', 'the', 'building', 'blocks', 'of', 'specialty',
  'products', '.', '"', 'this', 'kind', 'of', 'commodity', 'chemical', 'business', 'will', 'never',
  'be', 'a', 'glamorous', ',', 'high', '-', 'margin', 'business', ',"', 'cain', 'said', ',',
  'adding', 'that', 'demand', 'is', 'expected', 'to', 'grow', 'by', 'about', 'three', 'pct',
  'annually', '.', 'garo', 'armen', ',', 'an', 'analyst', 'with', 'dean', 'witter', 'reynolds', ',',
  'said', 'chemical', 'makers', 'have', 'also', 'benefitted', 'by', 'increasing', 'demand', 'for',
  'plastics', 'as', 'prices', 'become', 'more', 'competitive', 'with', 'aluminum', ',', 'wood',
  'and', 'steel', 'products', '.', 'armen', 'estimated', 'the', 'upturn', 'in', 'the', 'chemical',
  'business', 'could', 'last', 'as', 'long', 'as', 'four', 'or', 'five', 'years', ',', 'provided',
  'the', 'u', '.', 's', '.', 'economy', 'continues', 'its', 'modest', 'rate', 'of', 'growth', '.',
  '<END>'],
 ['<START>', 'turkey', 'calls', 'for', 'dialogue', 'to', 'solve', 'dispute', 'turkey', 'said',
  'today', 'its', 'disputes', 'with', 'greece', ',', 'including', 'rights', 'on', 'the',
  'continental', 'shelf', 'in', 'the', 'aegean', 'sea', ',', 'should', 'be', 'solved', 'through',
  'negotiations', '.', 'a', 'foreign', 'ministry', 'statement', 'said', 'the', 'latest', 'crisis',
  'between', 'the', 'two', 'nato', 'members', 'stemmed', 'from', 'the', 'continental', 'shelf',
  'dispute', 'and', 'an', 'agreement', 'on', 'this', 'issue', 'would', 'effect', 'the', 'security',
  ',', 'economy', 'and', 'other', 'rights', 'of', 'both', 'countries', '.', '"', 'as', 'the',
  'issue', 'is', 'basicly', 'political', ',', 'a', 'solution', 'can', 'only', 'be', 'found', 'by',
  'bilateral', 'negotiations', ',"', 'the', 'statement', 'said', '.', 'greece', 'has', 'repeatedly',
  'said', 'the', 'issue', 'was', 'legal', 'and', 'could', 'be', 'solved', 'at', 'the',
  'international', 'court', 'of', 'justice', '.', 'the', 'two', 'countries', 'approached', 'armed',
  'confrontation', 'last', 'month', 'after', 'greece', 'announced', 'it', 'planned', 'oil',
  'exploration', 'work', 'in', 'the', 'aegean', 'and', 'turkey', 'said', 'it', 'would', 'also',
  'search', 'for', 'oil', '.', 'a', 'face', '-', 'off', 'was', 'averted', 'when', 'turkey',
  'confined', 'its', 'research', 'to', 'territorrial', 'waters', '.', '"', 'the', 'latest',
  'crises', 'created', 'an', 'historic', 'opportunity', 'to', 'solve', 'the', 'disputes', 'between',
  'the', 'two', 'countries', ',"', 'the', 'foreign', 'ministry', 'statement', 'said', '.', 'turkey',
  "'", 's', 'ambassador', 'in', 'athens', ',', 'nazmi', 'akiman', ',', 'was', 'due', 'to', 'meet',
  'prime', 'minister', 'andreas', 'papandreou', 'today', 'for', 'the', 'greek', 'reply', 'to', 'a',
  'message', 'sent', 'last', 'week', 'by', 'turkish', 'prime', 'minister', 'turgut', 'ozal', '.',
  'the', 'contents', 'of', 'the', 'message', 'were', 'not', 'disclosed', '.', '<END>']]

    \end{Verbatim}

    \hypertarget{question-1.1-implement-distinct_words-code-2-points}{%
\subsubsection{\texorpdfstring{Question 1.1: Implement
\texttt{distinct\_words} {[}code{]} (2
points)}{Question 1.1: Implement distinct\_words {[}code{]} (2 points)}}\label{question-1.1-implement-distinct_words-code-2-points}}

Write a method to work out the distinct words (word types) that occur in
the corpus. You can do this with \texttt{for} loops, but it's more
efficient to do it with Python list comprehensions. In particular,
\href{https://coderwall.com/p/rcmaea/flatten-a-list-of-lists-in-one-line-in-python}{this}
may be useful to flatten a list of lists. If you're not familiar with
Python list comprehensions in general, here's
\href{https://python-3-patterns-idioms-test.readthedocs.io/en/latest/Comprehensions.html}{more
information}.

You may find it useful to use
\href{https://www.w3schools.com/python/python_sets.asp}{Python sets} to
remove duplicate words.

    \begin{Verbatim}[commandchars=\\\{\}]
{\color{incolor}In [{\color{incolor}4}]:} \PY{k}{def} \PY{n+nf}{distinct\PYZus{}words}\PY{p}{(}\PY{n}{corpus}\PY{p}{)}\PY{p}{:}
            \PY{l+s+sd}{\PYZdq{}\PYZdq{}\PYZdq{} Determine a list of distinct words for the corpus.}
        \PY{l+s+sd}{        Params:}
        \PY{l+s+sd}{            corpus (list of list of strings): corpus of documents}
        \PY{l+s+sd}{        Return:}
        \PY{l+s+sd}{            corpus\PYZus{}words (list of strings): list of distinct words across the corpus, sorted (using python \PYZsq{}sorted\PYZsq{} function)}
        \PY{l+s+sd}{            num\PYZus{}corpus\PYZus{}words (integer): number of distinct words across the corpus}
        \PY{l+s+sd}{    \PYZdq{}\PYZdq{}\PYZdq{}}
            \PY{n}{corpus\PYZus{}words} \PY{o}{=} \PY{p}{[}\PY{p}{]}
            \PY{n}{num\PYZus{}corpus\PYZus{}words} \PY{o}{=} \PY{o}{\PYZhy{}}\PY{l+m+mi}{1}
            
            \PY{c+c1}{\PYZsh{} \PYZhy{}\PYZhy{}\PYZhy{}\PYZhy{}\PYZhy{}\PYZhy{}\PYZhy{}\PYZhy{}\PYZhy{}\PYZhy{}\PYZhy{}\PYZhy{}\PYZhy{}\PYZhy{}\PYZhy{}\PYZhy{}\PYZhy{}\PYZhy{}}
            \PY{c+c1}{\PYZsh{} Write your implementation here.}
           
            \PY{n}{corpus\PYZus{}words} \PY{o}{=} \PY{p}{[}\PY{n}{word} \PY{k}{for} \PY{n}{line} \PY{o+ow}{in} \PY{n}{corpus} \PY{k}{for} \PY{n}{word} \PY{o+ow}{in} \PY{n}{line}\PY{p}{]}
            \PY{n}{corpus\PYZus{}words} \PY{o}{=} \PY{n+nb}{set}\PY{p}{(}\PY{n}{corpus\PYZus{}words}\PY{p}{)}
            \PY{n}{corpus\PYZus{}words} \PY{o}{=} \PY{n+nb}{sorted}\PY{p}{(}\PY{n+nb}{list}\PY{p}{(}\PY{n}{corpus\PYZus{}words}\PY{p}{)}\PY{p}{)}
            \PY{n}{num\PYZus{}corpus\PYZus{}words} \PY{o}{=} \PY{n+nb}{len}\PY{p}{(}\PY{n}{corpus\PYZus{}words}\PY{p}{)}
        
            \PY{c+c1}{\PYZsh{} \PYZhy{}\PYZhy{}\PYZhy{}\PYZhy{}\PYZhy{}\PYZhy{}\PYZhy{}\PYZhy{}\PYZhy{}\PYZhy{}\PYZhy{}\PYZhy{}\PYZhy{}\PYZhy{}\PYZhy{}\PYZhy{}\PYZhy{}\PYZhy{}}
        
            \PY{k}{return} \PY{n}{corpus\PYZus{}words}\PY{p}{,} \PY{n}{num\PYZus{}corpus\PYZus{}words}
\end{Verbatim}

    \begin{Verbatim}[commandchars=\\\{\}]
{\color{incolor}In [{\color{incolor}5}]:} \PY{c+c1}{\PYZsh{} \PYZhy{}\PYZhy{}\PYZhy{}\PYZhy{}\PYZhy{}\PYZhy{}\PYZhy{}\PYZhy{}\PYZhy{}\PYZhy{}\PYZhy{}\PYZhy{}\PYZhy{}\PYZhy{}\PYZhy{}\PYZhy{}\PYZhy{}\PYZhy{}\PYZhy{}\PYZhy{}\PYZhy{}}
        \PY{c+c1}{\PYZsh{} Run this sanity check}
        \PY{c+c1}{\PYZsh{} Note that this not an exhaustive check for correctness.}
        \PY{c+c1}{\PYZsh{} \PYZhy{}\PYZhy{}\PYZhy{}\PYZhy{}\PYZhy{}\PYZhy{}\PYZhy{}\PYZhy{}\PYZhy{}\PYZhy{}\PYZhy{}\PYZhy{}\PYZhy{}\PYZhy{}\PYZhy{}\PYZhy{}\PYZhy{}\PYZhy{}\PYZhy{}\PYZhy{}\PYZhy{}}
        
        \PY{c+c1}{\PYZsh{} Define toy corpus}
        \PY{n}{test\PYZus{}corpus} \PY{o}{=} \PY{p}{[}\PY{l+s+s2}{\PYZdq{}}\PY{l+s+si}{\PYZob{}\PYZcb{}}\PY{l+s+s2}{ All that glitters isn}\PY{l+s+s2}{\PYZsq{}}\PY{l+s+s2}{t gold }\PY{l+s+si}{\PYZob{}\PYZcb{}}\PY{l+s+s2}{\PYZdq{}}\PY{o}{.}\PY{n}{format}\PY{p}{(}\PY{n}{START\PYZus{}TOKEN}\PY{p}{,} \PY{n}{END\PYZus{}TOKEN}\PY{p}{)}\PY{o}{.}\PY{n}{split}\PY{p}{(}\PY{l+s+s2}{\PYZdq{}}\PY{l+s+s2}{ }\PY{l+s+s2}{\PYZdq{}}\PY{p}{)}\PY{p}{,} \PY{l+s+s2}{\PYZdq{}}\PY{l+s+si}{\PYZob{}\PYZcb{}}\PY{l+s+s2}{ All}\PY{l+s+s2}{\PYZsq{}}\PY{l+s+s2}{s well that ends well }\PY{l+s+si}{\PYZob{}\PYZcb{}}\PY{l+s+s2}{\PYZdq{}}\PY{o}{.}\PY{n}{format}\PY{p}{(}\PY{n}{START\PYZus{}TOKEN}\PY{p}{,} \PY{n}{END\PYZus{}TOKEN}\PY{p}{)}\PY{o}{.}\PY{n}{split}\PY{p}{(}\PY{l+s+s2}{\PYZdq{}}\PY{l+s+s2}{ }\PY{l+s+s2}{\PYZdq{}}\PY{p}{)}\PY{p}{]}
        \PY{n}{test\PYZus{}corpus\PYZus{}words}\PY{p}{,} \PY{n}{num\PYZus{}corpus\PYZus{}words} \PY{o}{=} \PY{n}{distinct\PYZus{}words}\PY{p}{(}\PY{n}{test\PYZus{}corpus}\PY{p}{)}
        
        \PY{c+c1}{\PYZsh{} Correct answers}
        \PY{n}{ans\PYZus{}test\PYZus{}corpus\PYZus{}words} \PY{o}{=} \PY{n+nb}{sorted}\PY{p}{(}\PY{p}{[}\PY{n}{START\PYZus{}TOKEN}\PY{p}{,} \PY{l+s+s2}{\PYZdq{}}\PY{l+s+s2}{All}\PY{l+s+s2}{\PYZdq{}}\PY{p}{,} \PY{l+s+s2}{\PYZdq{}}\PY{l+s+s2}{ends}\PY{l+s+s2}{\PYZdq{}}\PY{p}{,} \PY{l+s+s2}{\PYZdq{}}\PY{l+s+s2}{that}\PY{l+s+s2}{\PYZdq{}}\PY{p}{,} \PY{l+s+s2}{\PYZdq{}}\PY{l+s+s2}{gold}\PY{l+s+s2}{\PYZdq{}}\PY{p}{,} \PY{l+s+s2}{\PYZdq{}}\PY{l+s+s2}{All}\PY{l+s+s2}{\PYZsq{}}\PY{l+s+s2}{s}\PY{l+s+s2}{\PYZdq{}}\PY{p}{,} \PY{l+s+s2}{\PYZdq{}}\PY{l+s+s2}{glitters}\PY{l+s+s2}{\PYZdq{}}\PY{p}{,} \PY{l+s+s2}{\PYZdq{}}\PY{l+s+s2}{isn}\PY{l+s+s2}{\PYZsq{}}\PY{l+s+s2}{t}\PY{l+s+s2}{\PYZdq{}}\PY{p}{,} \PY{l+s+s2}{\PYZdq{}}\PY{l+s+s2}{well}\PY{l+s+s2}{\PYZdq{}}\PY{p}{,} \PY{n}{END\PYZus{}TOKEN}\PY{p}{]}\PY{p}{)}
        \PY{n}{ans\PYZus{}num\PYZus{}corpus\PYZus{}words} \PY{o}{=} \PY{n+nb}{len}\PY{p}{(}\PY{n}{ans\PYZus{}test\PYZus{}corpus\PYZus{}words}\PY{p}{)}
        
        \PY{c+c1}{\PYZsh{} Test correct number of words}
        \PY{k}{assert}\PY{p}{(}\PY{n}{num\PYZus{}corpus\PYZus{}words} \PY{o}{==} \PY{n}{ans\PYZus{}num\PYZus{}corpus\PYZus{}words}\PY{p}{)}\PY{p}{,} \PY{l+s+s2}{\PYZdq{}}\PY{l+s+s2}{Incorrect number of distinct words. Correct: }\PY{l+s+si}{\PYZob{}\PYZcb{}}\PY{l+s+s2}{. Yours: }\PY{l+s+si}{\PYZob{}\PYZcb{}}\PY{l+s+s2}{\PYZdq{}}\PY{o}{.}\PY{n}{format}\PY{p}{(}\PY{n}{ans\PYZus{}num\PYZus{}corpus\PYZus{}words}\PY{p}{,} \PY{n}{num\PYZus{}corpus\PYZus{}words}\PY{p}{)}
        
        \PY{c+c1}{\PYZsh{} Test correct words}
        \PY{k}{assert} \PY{p}{(}\PY{n}{test\PYZus{}corpus\PYZus{}words} \PY{o}{==} \PY{n}{ans\PYZus{}test\PYZus{}corpus\PYZus{}words}\PY{p}{)}\PY{p}{,} \PY{l+s+s2}{\PYZdq{}}\PY{l+s+s2}{Incorrect corpus\PYZus{}words.}\PY{l+s+se}{\PYZbs{}n}\PY{l+s+s2}{Correct: }\PY{l+s+si}{\PYZob{}\PYZcb{}}\PY{l+s+se}{\PYZbs{}n}\PY{l+s+s2}{Yours:   }\PY{l+s+si}{\PYZob{}\PYZcb{}}\PY{l+s+s2}{\PYZdq{}}\PY{o}{.}\PY{n}{format}\PY{p}{(}\PY{n+nb}{str}\PY{p}{(}\PY{n}{ans\PYZus{}test\PYZus{}corpus\PYZus{}words}\PY{p}{)}\PY{p}{,} \PY{n+nb}{str}\PY{p}{(}\PY{n}{test\PYZus{}corpus\PYZus{}words}\PY{p}{)}\PY{p}{)}
        
        \PY{c+c1}{\PYZsh{} Print Success}
        \PY{n+nb}{print} \PY{p}{(}\PY{l+s+s2}{\PYZdq{}}\PY{l+s+s2}{\PYZhy{}}\PY{l+s+s2}{\PYZdq{}} \PY{o}{*} \PY{l+m+mi}{80}\PY{p}{)}
        \PY{n+nb}{print}\PY{p}{(}\PY{l+s+s2}{\PYZdq{}}\PY{l+s+s2}{Passed All Tests!}\PY{l+s+s2}{\PYZdq{}}\PY{p}{)}
        \PY{n+nb}{print} \PY{p}{(}\PY{l+s+s2}{\PYZdq{}}\PY{l+s+s2}{\PYZhy{}}\PY{l+s+s2}{\PYZdq{}} \PY{o}{*} \PY{l+m+mi}{80}\PY{p}{)}
\end{Verbatim}

    \begin{Verbatim}[commandchars=\\\{\}]
--------------------------------------------------------------------------------
Passed All Tests!
--------------------------------------------------------------------------------

    \end{Verbatim}

    \hypertarget{question-1.2-implement-compute_co_occurrence_matrix-code-3-points}{%
\subsubsection{\texorpdfstring{Question 1.2: Implement
\texttt{compute\_co\_occurrence\_matrix} {[}code{]} (3
points)}{Question 1.2: Implement compute\_co\_occurrence\_matrix {[}code{]} (3 points)}}\label{question-1.2-implement-compute_co_occurrence_matrix-code-3-points}}

Write a method that constructs a co-occurrence matrix for a certain
window-size \(n\) (with a default of 4), considering words \(n\) before
and \(n\) after the word in the center of the window. Here, we start to
use \texttt{numpy\ (np)} to represent vectors, matrices, and tensors. If
you're not familiar with NumPy, there's a NumPy tutorial in the second
half of this cs231n
\href{http://cs231n.github.io/python-numpy-tutorial/}{Python NumPy
tutorial}.

    \begin{Verbatim}[commandchars=\\\{\}]
{\color{incolor}In [{\color{incolor}6}]:} \PY{k}{def} \PY{n+nf}{compute\PYZus{}co\PYZus{}occurrence\PYZus{}matrix}\PY{p}{(}\PY{n}{corpus}\PY{p}{,} \PY{n}{window\PYZus{}size}\PY{o}{=}\PY{l+m+mi}{4}\PY{p}{)}\PY{p}{:}
            \PY{l+s+sd}{\PYZdq{}\PYZdq{}\PYZdq{} Compute co\PYZhy{}occurrence matrix for the given corpus and window\PYZus{}size (default of 4).}
        \PY{l+s+sd}{    }
        \PY{l+s+sd}{        Note: Each word in a document should be at the center of a window. Words near edges will have a smaller}
        \PY{l+s+sd}{              number of co\PYZhy{}occurring words.}
        \PY{l+s+sd}{              }
        \PY{l+s+sd}{              For example, if we take the document \PYZdq{}\PYZlt{}START\PYZgt{} All that glitters is not gold \PYZlt{}END\PYZgt{}\PYZdq{} with window size of 4,}
        \PY{l+s+sd}{              \PYZdq{}All\PYZdq{} will co\PYZhy{}occur with \PYZdq{}\PYZlt{}START\PYZgt{}\PYZdq{}, \PYZdq{}that\PYZdq{}, \PYZdq{}glitters\PYZdq{}, \PYZdq{}is\PYZdq{}, and \PYZdq{}not\PYZdq{}.}
        \PY{l+s+sd}{    }
        \PY{l+s+sd}{        Params:}
        \PY{l+s+sd}{            corpus (list of list of strings): corpus of documents}
        \PY{l+s+sd}{            window\PYZus{}size (int): size of context window}
        \PY{l+s+sd}{        Return:}
        \PY{l+s+sd}{            M (a symmetric numpy matrix of shape (number of unique words in the corpus , number of unique words in the corpus)): }
        \PY{l+s+sd}{                Co\PYZhy{}occurence matrix of word counts. }
        \PY{l+s+sd}{                The ordering of the words in the rows/columns should be the same as the ordering of the words given by the distinct\PYZus{}words function.}
        \PY{l+s+sd}{            word2Ind (dict): dictionary that maps word to index (i.e. row/column number) for matrix M.}
        \PY{l+s+sd}{    \PYZdq{}\PYZdq{}\PYZdq{}}
            \PY{n}{words}\PY{p}{,} \PY{n}{num\PYZus{}words} \PY{o}{=} \PY{n}{distinct\PYZus{}words}\PY{p}{(}\PY{n}{corpus}\PY{p}{)}
            \PY{n}{M} \PY{o}{=} \PY{k+kc}{None}
            \PY{n}{word2Ind} \PY{o}{=} \PY{p}{\PYZob{}}\PY{p}{\PYZcb{}}
            
            \PY{c+c1}{\PYZsh{} \PYZhy{}\PYZhy{}\PYZhy{}\PYZhy{}\PYZhy{}\PYZhy{}\PYZhy{}\PYZhy{}\PYZhy{}\PYZhy{}\PYZhy{}\PYZhy{}\PYZhy{}\PYZhy{}\PYZhy{}\PYZhy{}\PYZhy{}\PYZhy{}}
            \PY{c+c1}{\PYZsh{} Write your implementation here.}
            
            \PY{n}{M} \PY{o}{=} \PY{n}{np}\PY{o}{.}\PY{n}{zeros}\PY{p}{(}\PY{p}{(}\PY{n}{num\PYZus{}words}\PY{p}{,} \PY{n}{num\PYZus{}words}\PY{p}{)}\PY{p}{)}
            \PY{n}{word2Ind} \PY{o}{=} \PY{p}{\PYZob{}}\PY{n}{w} \PY{p}{:} \PY{n}{i} \PY{k}{for} \PY{n}{i}\PY{p}{,} \PY{n}{w} \PY{o+ow}{in} \PY{n+nb}{enumerate}\PY{p}{(}\PY{n}{words}\PY{p}{)}\PY{p}{\PYZcb{}}
            
            \PY{k}{for} \PY{n}{d} \PY{o+ow}{in} \PY{n}{corpus}\PY{p}{:}
                \PY{k}{for} \PY{n}{i}\PY{p}{,} \PY{n}{w} \PY{o+ow}{in} \PY{n+nb}{enumerate}\PY{p}{(}\PY{n}{d}\PY{p}{)}\PY{p}{:}
                    \PY{k}{for} \PY{n}{j} \PY{o+ow}{in} \PY{n+nb}{range}\PY{p}{(}\PY{n}{i} \PY{o}{\PYZhy{}} \PY{n}{window\PYZus{}size}\PY{p}{,} \PY{n}{i} \PY{o}{+} \PY{n}{window\PYZus{}size} \PY{o}{+} \PY{l+m+mi}{1}\PY{p}{)}\PY{p}{:}
                        \PY{k}{if} \PY{n}{j} \PY{o}{\PYZlt{}} \PY{l+m+mi}{0} \PY{o+ow}{or} \PY{n}{j} \PY{o}{\PYZgt{}}\PY{o}{=} \PY{n+nb}{len}\PY{p}{(}\PY{n}{d}\PY{p}{)}\PY{p}{:}
                            \PY{k}{continue}
                        \PY{k}{if} \PY{n}{j} \PY{o}{!=} \PY{n}{i}\PY{p}{:}
                            \PY{n}{M}\PY{p}{[}\PY{n}{word2Ind}\PY{p}{[}\PY{n}{w}\PY{p}{]}\PY{p}{,} \PY{n}{word2Ind}\PY{p}{[}\PY{n}{d}\PY{p}{[}\PY{n}{j}\PY{p}{]}\PY{p}{]}\PY{p}{]} \PY{o}{+}\PY{o}{=} \PY{l+m+mi}{1}
        
            \PY{c+c1}{\PYZsh{} \PYZhy{}\PYZhy{}\PYZhy{}\PYZhy{}\PYZhy{}\PYZhy{}\PYZhy{}\PYZhy{}\PYZhy{}\PYZhy{}\PYZhy{}\PYZhy{}\PYZhy{}\PYZhy{}\PYZhy{}\PYZhy{}\PYZhy{}\PYZhy{}}
            \PY{k}{return} \PY{n}{M}\PY{p}{,} \PY{n}{word2Ind}
\end{Verbatim}

    \begin{Verbatim}[commandchars=\\\{\}]
{\color{incolor}In [{\color{incolor}7}]:} \PY{c+c1}{\PYZsh{} \PYZhy{}\PYZhy{}\PYZhy{}\PYZhy{}\PYZhy{}\PYZhy{}\PYZhy{}\PYZhy{}\PYZhy{}\PYZhy{}\PYZhy{}\PYZhy{}\PYZhy{}\PYZhy{}\PYZhy{}\PYZhy{}\PYZhy{}\PYZhy{}\PYZhy{}\PYZhy{}\PYZhy{}}
        \PY{c+c1}{\PYZsh{} Run this sanity check}
        \PY{c+c1}{\PYZsh{} Note that this is not an exhaustive check for correctness.}
        \PY{c+c1}{\PYZsh{} \PYZhy{}\PYZhy{}\PYZhy{}\PYZhy{}\PYZhy{}\PYZhy{}\PYZhy{}\PYZhy{}\PYZhy{}\PYZhy{}\PYZhy{}\PYZhy{}\PYZhy{}\PYZhy{}\PYZhy{}\PYZhy{}\PYZhy{}\PYZhy{}\PYZhy{}\PYZhy{}\PYZhy{}}
        
        \PY{c+c1}{\PYZsh{} Define toy corpus and get student\PYZsq{}s co\PYZhy{}occurrence matrix}
        \PY{n}{test\PYZus{}corpus} \PY{o}{=} \PY{p}{[}\PY{l+s+s2}{\PYZdq{}}\PY{l+s+si}{\PYZob{}\PYZcb{}}\PY{l+s+s2}{ All that glitters isn}\PY{l+s+s2}{\PYZsq{}}\PY{l+s+s2}{t gold }\PY{l+s+si}{\PYZob{}\PYZcb{}}\PY{l+s+s2}{\PYZdq{}}\PY{o}{.}\PY{n}{format}\PY{p}{(}\PY{n}{START\PYZus{}TOKEN}\PY{p}{,} \PY{n}{END\PYZus{}TOKEN}\PY{p}{)}\PY{o}{.}\PY{n}{split}\PY{p}{(}\PY{l+s+s2}{\PYZdq{}}\PY{l+s+s2}{ }\PY{l+s+s2}{\PYZdq{}}\PY{p}{)}\PY{p}{,} \PY{l+s+s2}{\PYZdq{}}\PY{l+s+si}{\PYZob{}\PYZcb{}}\PY{l+s+s2}{ All}\PY{l+s+s2}{\PYZsq{}}\PY{l+s+s2}{s well that ends well }\PY{l+s+si}{\PYZob{}\PYZcb{}}\PY{l+s+s2}{\PYZdq{}}\PY{o}{.}\PY{n}{format}\PY{p}{(}\PY{n}{START\PYZus{}TOKEN}\PY{p}{,} \PY{n}{END\PYZus{}TOKEN}\PY{p}{)}\PY{o}{.}\PY{n}{split}\PY{p}{(}\PY{l+s+s2}{\PYZdq{}}\PY{l+s+s2}{ }\PY{l+s+s2}{\PYZdq{}}\PY{p}{)}\PY{p}{]}
        \PY{n}{M\PYZus{}test}\PY{p}{,} \PY{n}{word2Ind\PYZus{}test} \PY{o}{=} \PY{n}{compute\PYZus{}co\PYZus{}occurrence\PYZus{}matrix}\PY{p}{(}\PY{n}{test\PYZus{}corpus}\PY{p}{,} \PY{n}{window\PYZus{}size}\PY{o}{=}\PY{l+m+mi}{1}\PY{p}{)}
        
        \PY{c+c1}{\PYZsh{} Correct M and word2Ind}
        \PY{n}{M\PYZus{}test\PYZus{}ans} \PY{o}{=} \PY{n}{np}\PY{o}{.}\PY{n}{array}\PY{p}{(} 
            \PY{p}{[}\PY{p}{[}\PY{l+m+mf}{0.}\PY{p}{,} \PY{l+m+mf}{0.}\PY{p}{,} \PY{l+m+mf}{0.}\PY{p}{,} \PY{l+m+mf}{0.}\PY{p}{,} \PY{l+m+mf}{0.}\PY{p}{,} \PY{l+m+mf}{0.}\PY{p}{,} \PY{l+m+mf}{1.}\PY{p}{,} \PY{l+m+mf}{0.}\PY{p}{,} \PY{l+m+mf}{0.}\PY{p}{,} \PY{l+m+mf}{1.}\PY{p}{,}\PY{p}{]}\PY{p}{,}
             \PY{p}{[}\PY{l+m+mf}{0.}\PY{p}{,} \PY{l+m+mf}{0.}\PY{p}{,} \PY{l+m+mf}{1.}\PY{p}{,} \PY{l+m+mf}{1.}\PY{p}{,} \PY{l+m+mf}{0.}\PY{p}{,} \PY{l+m+mf}{0.}\PY{p}{,} \PY{l+m+mf}{0.}\PY{p}{,} \PY{l+m+mf}{0.}\PY{p}{,} \PY{l+m+mf}{0.}\PY{p}{,} \PY{l+m+mf}{0.}\PY{p}{,}\PY{p}{]}\PY{p}{,}
             \PY{p}{[}\PY{l+m+mf}{0.}\PY{p}{,} \PY{l+m+mf}{1.}\PY{p}{,} \PY{l+m+mf}{0.}\PY{p}{,} \PY{l+m+mf}{0.}\PY{p}{,} \PY{l+m+mf}{0.}\PY{p}{,} \PY{l+m+mf}{0.}\PY{p}{,} \PY{l+m+mf}{0.}\PY{p}{,} \PY{l+m+mf}{0.}\PY{p}{,} \PY{l+m+mf}{1.}\PY{p}{,} \PY{l+m+mf}{0.}\PY{p}{,}\PY{p}{]}\PY{p}{,}
             \PY{p}{[}\PY{l+m+mf}{0.}\PY{p}{,} \PY{l+m+mf}{1.}\PY{p}{,} \PY{l+m+mf}{0.}\PY{p}{,} \PY{l+m+mf}{0.}\PY{p}{,} \PY{l+m+mf}{0.}\PY{p}{,} \PY{l+m+mf}{0.}\PY{p}{,} \PY{l+m+mf}{0.}\PY{p}{,} \PY{l+m+mf}{0.}\PY{p}{,} \PY{l+m+mf}{0.}\PY{p}{,} \PY{l+m+mf}{1.}\PY{p}{,}\PY{p}{]}\PY{p}{,}
             \PY{p}{[}\PY{l+m+mf}{0.}\PY{p}{,} \PY{l+m+mf}{0.}\PY{p}{,} \PY{l+m+mf}{0.}\PY{p}{,} \PY{l+m+mf}{0.}\PY{p}{,} \PY{l+m+mf}{0.}\PY{p}{,} \PY{l+m+mf}{0.}\PY{p}{,} \PY{l+m+mf}{0.}\PY{p}{,} \PY{l+m+mf}{0.}\PY{p}{,} \PY{l+m+mf}{1.}\PY{p}{,} \PY{l+m+mf}{1.}\PY{p}{,}\PY{p}{]}\PY{p}{,}
             \PY{p}{[}\PY{l+m+mf}{0.}\PY{p}{,} \PY{l+m+mf}{0.}\PY{p}{,} \PY{l+m+mf}{0.}\PY{p}{,} \PY{l+m+mf}{0.}\PY{p}{,} \PY{l+m+mf}{0.}\PY{p}{,} \PY{l+m+mf}{0.}\PY{p}{,} \PY{l+m+mf}{0.}\PY{p}{,} \PY{l+m+mf}{1.}\PY{p}{,} \PY{l+m+mf}{1.}\PY{p}{,} \PY{l+m+mf}{0.}\PY{p}{,}\PY{p}{]}\PY{p}{,}
             \PY{p}{[}\PY{l+m+mf}{1.}\PY{p}{,} \PY{l+m+mf}{0.}\PY{p}{,} \PY{l+m+mf}{0.}\PY{p}{,} \PY{l+m+mf}{0.}\PY{p}{,} \PY{l+m+mf}{0.}\PY{p}{,} \PY{l+m+mf}{0.}\PY{p}{,} \PY{l+m+mf}{0.}\PY{p}{,} \PY{l+m+mf}{1.}\PY{p}{,} \PY{l+m+mf}{0.}\PY{p}{,} \PY{l+m+mf}{0.}\PY{p}{,}\PY{p}{]}\PY{p}{,}
             \PY{p}{[}\PY{l+m+mf}{0.}\PY{p}{,} \PY{l+m+mf}{0.}\PY{p}{,} \PY{l+m+mf}{0.}\PY{p}{,} \PY{l+m+mf}{0.}\PY{p}{,} \PY{l+m+mf}{0.}\PY{p}{,} \PY{l+m+mf}{1.}\PY{p}{,} \PY{l+m+mf}{1.}\PY{p}{,} \PY{l+m+mf}{0.}\PY{p}{,} \PY{l+m+mf}{0.}\PY{p}{,} \PY{l+m+mf}{0.}\PY{p}{,}\PY{p}{]}\PY{p}{,}
             \PY{p}{[}\PY{l+m+mf}{0.}\PY{p}{,} \PY{l+m+mf}{0.}\PY{p}{,} \PY{l+m+mf}{1.}\PY{p}{,} \PY{l+m+mf}{0.}\PY{p}{,} \PY{l+m+mf}{1.}\PY{p}{,} \PY{l+m+mf}{1.}\PY{p}{,} \PY{l+m+mf}{0.}\PY{p}{,} \PY{l+m+mf}{0.}\PY{p}{,} \PY{l+m+mf}{0.}\PY{p}{,} \PY{l+m+mf}{1.}\PY{p}{,}\PY{p}{]}\PY{p}{,}
             \PY{p}{[}\PY{l+m+mf}{1.}\PY{p}{,} \PY{l+m+mf}{0.}\PY{p}{,} \PY{l+m+mf}{0.}\PY{p}{,} \PY{l+m+mf}{1.}\PY{p}{,} \PY{l+m+mf}{1.}\PY{p}{,} \PY{l+m+mf}{0.}\PY{p}{,} \PY{l+m+mf}{0.}\PY{p}{,} \PY{l+m+mf}{0.}\PY{p}{,} \PY{l+m+mf}{1.}\PY{p}{,} \PY{l+m+mf}{0.}\PY{p}{,}\PY{p}{]}\PY{p}{]}
        \PY{p}{)}
        \PY{n}{ans\PYZus{}test\PYZus{}corpus\PYZus{}words} \PY{o}{=} \PY{n+nb}{sorted}\PY{p}{(}\PY{p}{[}\PY{n}{START\PYZus{}TOKEN}\PY{p}{,} \PY{l+s+s2}{\PYZdq{}}\PY{l+s+s2}{All}\PY{l+s+s2}{\PYZdq{}}\PY{p}{,} \PY{l+s+s2}{\PYZdq{}}\PY{l+s+s2}{ends}\PY{l+s+s2}{\PYZdq{}}\PY{p}{,} \PY{l+s+s2}{\PYZdq{}}\PY{l+s+s2}{that}\PY{l+s+s2}{\PYZdq{}}\PY{p}{,} \PY{l+s+s2}{\PYZdq{}}\PY{l+s+s2}{gold}\PY{l+s+s2}{\PYZdq{}}\PY{p}{,} \PY{l+s+s2}{\PYZdq{}}\PY{l+s+s2}{All}\PY{l+s+s2}{\PYZsq{}}\PY{l+s+s2}{s}\PY{l+s+s2}{\PYZdq{}}\PY{p}{,} \PY{l+s+s2}{\PYZdq{}}\PY{l+s+s2}{glitters}\PY{l+s+s2}{\PYZdq{}}\PY{p}{,} \PY{l+s+s2}{\PYZdq{}}\PY{l+s+s2}{isn}\PY{l+s+s2}{\PYZsq{}}\PY{l+s+s2}{t}\PY{l+s+s2}{\PYZdq{}}\PY{p}{,} \PY{l+s+s2}{\PYZdq{}}\PY{l+s+s2}{well}\PY{l+s+s2}{\PYZdq{}}\PY{p}{,} \PY{n}{END\PYZus{}TOKEN}\PY{p}{]}\PY{p}{)}
        \PY{n}{word2Ind\PYZus{}ans} \PY{o}{=} \PY{n+nb}{dict}\PY{p}{(}\PY{n+nb}{zip}\PY{p}{(}\PY{n}{ans\PYZus{}test\PYZus{}corpus\PYZus{}words}\PY{p}{,} \PY{n+nb}{range}\PY{p}{(}\PY{n+nb}{len}\PY{p}{(}\PY{n}{ans\PYZus{}test\PYZus{}corpus\PYZus{}words}\PY{p}{)}\PY{p}{)}\PY{p}{)}\PY{p}{)}
        
        \PY{c+c1}{\PYZsh{} Test correct word2Ind}
        \PY{k}{assert} \PY{p}{(}\PY{n}{word2Ind\PYZus{}ans} \PY{o}{==} \PY{n}{word2Ind\PYZus{}test}\PY{p}{)}\PY{p}{,} \PY{l+s+s2}{\PYZdq{}}\PY{l+s+s2}{Your word2Ind is incorrect:}\PY{l+s+se}{\PYZbs{}n}\PY{l+s+s2}{Correct: }\PY{l+s+si}{\PYZob{}\PYZcb{}}\PY{l+s+se}{\PYZbs{}n}\PY{l+s+s2}{Yours: }\PY{l+s+si}{\PYZob{}\PYZcb{}}\PY{l+s+s2}{\PYZdq{}}\PY{o}{.}\PY{n}{format}\PY{p}{(}\PY{n}{word2Ind\PYZus{}ans}\PY{p}{,} \PY{n}{word2Ind\PYZus{}test}\PY{p}{)}
        
        \PY{c+c1}{\PYZsh{} Test correct M shape}
        \PY{k}{assert} \PY{p}{(}\PY{n}{M\PYZus{}test}\PY{o}{.}\PY{n}{shape} \PY{o}{==} \PY{n}{M\PYZus{}test\PYZus{}ans}\PY{o}{.}\PY{n}{shape}\PY{p}{)}\PY{p}{,} \PY{l+s+s2}{\PYZdq{}}\PY{l+s+s2}{M matrix has incorrect shape.}\PY{l+s+se}{\PYZbs{}n}\PY{l+s+s2}{Correct: }\PY{l+s+si}{\PYZob{}\PYZcb{}}\PY{l+s+se}{\PYZbs{}n}\PY{l+s+s2}{Yours: }\PY{l+s+si}{\PYZob{}\PYZcb{}}\PY{l+s+s2}{\PYZdq{}}\PY{o}{.}\PY{n}{format}\PY{p}{(}\PY{n}{M\PYZus{}test}\PY{o}{.}\PY{n}{shape}\PY{p}{,} \PY{n}{M\PYZus{}test\PYZus{}ans}\PY{o}{.}\PY{n}{shape}\PY{p}{)}
        
        \PY{c+c1}{\PYZsh{} Test correct M values}
        \PY{k}{for} \PY{n}{w1} \PY{o+ow}{in} \PY{n}{word2Ind\PYZus{}ans}\PY{o}{.}\PY{n}{keys}\PY{p}{(}\PY{p}{)}\PY{p}{:}
            \PY{n}{idx1} \PY{o}{=} \PY{n}{word2Ind\PYZus{}ans}\PY{p}{[}\PY{n}{w1}\PY{p}{]}
            \PY{k}{for} \PY{n}{w2} \PY{o+ow}{in} \PY{n}{word2Ind\PYZus{}ans}\PY{o}{.}\PY{n}{keys}\PY{p}{(}\PY{p}{)}\PY{p}{:}
                \PY{n}{idx2} \PY{o}{=} \PY{n}{word2Ind\PYZus{}ans}\PY{p}{[}\PY{n}{w2}\PY{p}{]}
                \PY{n}{student} \PY{o}{=} \PY{n}{M\PYZus{}test}\PY{p}{[}\PY{n}{idx1}\PY{p}{,} \PY{n}{idx2}\PY{p}{]}
                \PY{n}{correct} \PY{o}{=} \PY{n}{M\PYZus{}test\PYZus{}ans}\PY{p}{[}\PY{n}{idx1}\PY{p}{,} \PY{n}{idx2}\PY{p}{]}
                \PY{k}{if} \PY{n}{student} \PY{o}{!=} \PY{n}{correct}\PY{p}{:}
                    \PY{n+nb}{print}\PY{p}{(}\PY{l+s+s2}{\PYZdq{}}\PY{l+s+s2}{Correct M:}\PY{l+s+s2}{\PYZdq{}}\PY{p}{)}
                    \PY{n+nb}{print}\PY{p}{(}\PY{n}{M\PYZus{}test\PYZus{}ans}\PY{p}{)}
                    \PY{n+nb}{print}\PY{p}{(}\PY{l+s+s2}{\PYZdq{}}\PY{l+s+s2}{Your M: }\PY{l+s+s2}{\PYZdq{}}\PY{p}{)}
                    \PY{n+nb}{print}\PY{p}{(}\PY{n}{M\PYZus{}test}\PY{p}{)}
                    \PY{k}{raise} \PY{n+ne}{AssertionError}\PY{p}{(}\PY{l+s+s2}{\PYZdq{}}\PY{l+s+s2}{Incorrect count at index (}\PY{l+s+si}{\PYZob{}\PYZcb{}}\PY{l+s+s2}{, }\PY{l+s+si}{\PYZob{}\PYZcb{}}\PY{l+s+s2}{)=(}\PY{l+s+si}{\PYZob{}\PYZcb{}}\PY{l+s+s2}{, }\PY{l+s+si}{\PYZob{}\PYZcb{}}\PY{l+s+s2}{) in matrix M. Yours has }\PY{l+s+si}{\PYZob{}\PYZcb{}}\PY{l+s+s2}{ but should have }\PY{l+s+si}{\PYZob{}\PYZcb{}}\PY{l+s+s2}{.}\PY{l+s+s2}{\PYZdq{}}\PY{o}{.}\PY{n}{format}\PY{p}{(}\PY{n}{idx1}\PY{p}{,} \PY{n}{idx2}\PY{p}{,} \PY{n}{w1}\PY{p}{,} \PY{n}{w2}\PY{p}{,} \PY{n}{student}\PY{p}{,} \PY{n}{correct}\PY{p}{)}\PY{p}{)}
        
        \PY{c+c1}{\PYZsh{} Print Success}
        \PY{n+nb}{print} \PY{p}{(}\PY{l+s+s2}{\PYZdq{}}\PY{l+s+s2}{\PYZhy{}}\PY{l+s+s2}{\PYZdq{}} \PY{o}{*} \PY{l+m+mi}{80}\PY{p}{)}
        \PY{n+nb}{print}\PY{p}{(}\PY{l+s+s2}{\PYZdq{}}\PY{l+s+s2}{Passed All Tests!}\PY{l+s+s2}{\PYZdq{}}\PY{p}{)}
        \PY{n+nb}{print} \PY{p}{(}\PY{l+s+s2}{\PYZdq{}}\PY{l+s+s2}{\PYZhy{}}\PY{l+s+s2}{\PYZdq{}} \PY{o}{*} \PY{l+m+mi}{80}\PY{p}{)}
\end{Verbatim}

    \begin{Verbatim}[commandchars=\\\{\}]
--------------------------------------------------------------------------------
Passed All Tests!
--------------------------------------------------------------------------------

    \end{Verbatim}

    \hypertarget{question-1.3-implement-reduce_to_k_dim-code-1-point}{%
\subsubsection{\texorpdfstring{Question 1.3: Implement
\texttt{reduce\_to\_k\_dim} {[}code{]} (1
point)}{Question 1.3: Implement reduce\_to\_k\_dim {[}code{]} (1 point)}}\label{question-1.3-implement-reduce_to_k_dim-code-1-point}}

Construct a method that performs dimensionality reduction on the matrix
to produce k-dimensional embeddings. Use SVD to take the top k
components and produce a new matrix of k-dimensional embeddings.

\textbf{Note:} All of numpy, scipy, and scikit-learn (\texttt{sklearn})
provide \emph{some} implementation of SVD, but only scipy and sklearn
provide an implementation of Truncated SVD, and only sklearn provides an
efficient randomized algorithm for calculating large-scale Truncated
SVD. So please use
\href{https://scikit-learn.org/stable/modules/generated/sklearn.decomposition.TruncatedSVD.html}{sklearn.decomposition.TruncatedSVD}.

    \begin{Verbatim}[commandchars=\\\{\}]
{\color{incolor}In [{\color{incolor}8}]:} \PY{k}{def} \PY{n+nf}{reduce\PYZus{}to\PYZus{}k\PYZus{}dim}\PY{p}{(}\PY{n}{M}\PY{p}{,} \PY{n}{k}\PY{o}{=}\PY{l+m+mi}{2}\PY{p}{)}\PY{p}{:}
            \PY{l+s+sd}{\PYZdq{}\PYZdq{}\PYZdq{} Reduce a co\PYZhy{}occurence count matrix of dimensionality (num\PYZus{}corpus\PYZus{}words, num\PYZus{}corpus\PYZus{}words)}
        \PY{l+s+sd}{        to a matrix of dimensionality (num\PYZus{}corpus\PYZus{}words, k) using the following SVD function from Scikit\PYZhy{}Learn:}
        \PY{l+s+sd}{            \PYZhy{} http://scikit\PYZhy{}learn.org/stable/modules/generated/sklearn.decomposition.TruncatedSVD.html}
        \PY{l+s+sd}{    }
        \PY{l+s+sd}{        Params:}
        \PY{l+s+sd}{            M (numpy matrix of shape (number of unique words in the corpus , number of unique words in the corpus)): co\PYZhy{}occurence matrix of word counts}
        \PY{l+s+sd}{            k (int): embedding size of each word after dimension reduction}
        \PY{l+s+sd}{        Return:}
        \PY{l+s+sd}{            M\PYZus{}reduced (numpy matrix of shape (number of corpus words, k)): matrix of k\PYZhy{}dimensioal word embeddings.}
        \PY{l+s+sd}{                    In terms of the SVD from math class, this actually returns U * S}
        \PY{l+s+sd}{    \PYZdq{}\PYZdq{}\PYZdq{}}    
            \PY{n}{n\PYZus{}iters} \PY{o}{=} \PY{l+m+mi}{10}     \PY{c+c1}{\PYZsh{} Use this parameter in your call to `TruncatedSVD`}
            \PY{n}{M\PYZus{}reduced} \PY{o}{=} \PY{k+kc}{None}
            \PY{n+nb}{print}\PY{p}{(}\PY{l+s+s2}{\PYZdq{}}\PY{l+s+s2}{Running Truncated SVD over }\PY{l+s+si}{\PYZpc{}i}\PY{l+s+s2}{ words...}\PY{l+s+s2}{\PYZdq{}} \PY{o}{\PYZpc{}} \PY{p}{(}\PY{n}{M}\PY{o}{.}\PY{n}{shape}\PY{p}{[}\PY{l+m+mi}{0}\PY{p}{]}\PY{p}{)}\PY{p}{)}
            
                \PY{c+c1}{\PYZsh{} \PYZhy{}\PYZhy{}\PYZhy{}\PYZhy{}\PYZhy{}\PYZhy{}\PYZhy{}\PYZhy{}\PYZhy{}\PYZhy{}\PYZhy{}\PYZhy{}\PYZhy{}\PYZhy{}\PYZhy{}\PYZhy{}\PYZhy{}\PYZhy{}}
                \PY{c+c1}{\PYZsh{} Write your implementation here.}
                
            \PY{n}{svd} \PY{o}{=} \PY{n}{TruncatedSVD}\PY{p}{(}\PY{n}{n\PYZus{}components} \PY{o}{=} \PY{n}{k}\PY{p}{,} \PY{n}{n\PYZus{}iter} \PY{o}{=} \PY{n}{n\PYZus{}iters}\PY{p}{)}
            \PY{n}{M\PYZus{}reduced} \PY{o}{=} \PY{n}{svd}\PY{o}{.}\PY{n}{fit\PYZus{}transform}\PY{p}{(}\PY{n}{M}\PY{p}{)}
            
                \PY{c+c1}{\PYZsh{} \PYZhy{}\PYZhy{}\PYZhy{}\PYZhy{}\PYZhy{}\PYZhy{}\PYZhy{}\PYZhy{}\PYZhy{}\PYZhy{}\PYZhy{}\PYZhy{}\PYZhy{}\PYZhy{}\PYZhy{}\PYZhy{}\PYZhy{}\PYZhy{}}
        
            \PY{n+nb}{print}\PY{p}{(}\PY{l+s+s2}{\PYZdq{}}\PY{l+s+s2}{Done.}\PY{l+s+s2}{\PYZdq{}}\PY{p}{)}
            \PY{k}{return} \PY{n}{M\PYZus{}reduced}
\end{Verbatim}

    \begin{Verbatim}[commandchars=\\\{\}]
{\color{incolor}In [{\color{incolor}9}]:} \PY{c+c1}{\PYZsh{} \PYZhy{}\PYZhy{}\PYZhy{}\PYZhy{}\PYZhy{}\PYZhy{}\PYZhy{}\PYZhy{}\PYZhy{}\PYZhy{}\PYZhy{}\PYZhy{}\PYZhy{}\PYZhy{}\PYZhy{}\PYZhy{}\PYZhy{}\PYZhy{}\PYZhy{}\PYZhy{}\PYZhy{}}
        \PY{c+c1}{\PYZsh{} Run this sanity check}
        \PY{c+c1}{\PYZsh{} Note that this is not an exhaustive check for correctness }
        \PY{c+c1}{\PYZsh{} In fact we only check that your M\PYZus{}reduced has the right dimensions.}
        \PY{c+c1}{\PYZsh{} \PYZhy{}\PYZhy{}\PYZhy{}\PYZhy{}\PYZhy{}\PYZhy{}\PYZhy{}\PYZhy{}\PYZhy{}\PYZhy{}\PYZhy{}\PYZhy{}\PYZhy{}\PYZhy{}\PYZhy{}\PYZhy{}\PYZhy{}\PYZhy{}\PYZhy{}\PYZhy{}\PYZhy{}}
        
        \PY{c+c1}{\PYZsh{} Define toy corpus and run student code}
        \PY{n}{test\PYZus{}corpus} \PY{o}{=} \PY{p}{[}\PY{l+s+s2}{\PYZdq{}}\PY{l+s+si}{\PYZob{}\PYZcb{}}\PY{l+s+s2}{ All that glitters isn}\PY{l+s+s2}{\PYZsq{}}\PY{l+s+s2}{t gold }\PY{l+s+si}{\PYZob{}\PYZcb{}}\PY{l+s+s2}{\PYZdq{}}\PY{o}{.}\PY{n}{format}\PY{p}{(}\PY{n}{START\PYZus{}TOKEN}\PY{p}{,} \PY{n}{END\PYZus{}TOKEN}\PY{p}{)}\PY{o}{.}\PY{n}{split}\PY{p}{(}\PY{l+s+s2}{\PYZdq{}}\PY{l+s+s2}{ }\PY{l+s+s2}{\PYZdq{}}\PY{p}{)}\PY{p}{,} \PY{l+s+s2}{\PYZdq{}}\PY{l+s+si}{\PYZob{}\PYZcb{}}\PY{l+s+s2}{ All}\PY{l+s+s2}{\PYZsq{}}\PY{l+s+s2}{s well that ends well }\PY{l+s+si}{\PYZob{}\PYZcb{}}\PY{l+s+s2}{\PYZdq{}}\PY{o}{.}\PY{n}{format}\PY{p}{(}\PY{n}{START\PYZus{}TOKEN}\PY{p}{,} \PY{n}{END\PYZus{}TOKEN}\PY{p}{)}\PY{o}{.}\PY{n}{split}\PY{p}{(}\PY{l+s+s2}{\PYZdq{}}\PY{l+s+s2}{ }\PY{l+s+s2}{\PYZdq{}}\PY{p}{)}\PY{p}{]}
        \PY{n}{M\PYZus{}test}\PY{p}{,} \PY{n}{word2Ind\PYZus{}test} \PY{o}{=} \PY{n}{compute\PYZus{}co\PYZus{}occurrence\PYZus{}matrix}\PY{p}{(}\PY{n}{test\PYZus{}corpus}\PY{p}{,} \PY{n}{window\PYZus{}size}\PY{o}{=}\PY{l+m+mi}{1}\PY{p}{)}
        \PY{n}{M\PYZus{}test\PYZus{}reduced} \PY{o}{=} \PY{n}{reduce\PYZus{}to\PYZus{}k\PYZus{}dim}\PY{p}{(}\PY{n}{M\PYZus{}test}\PY{p}{,} \PY{n}{k}\PY{o}{=}\PY{l+m+mi}{2}\PY{p}{)}
        
        \PY{c+c1}{\PYZsh{} Test proper dimensions}
        \PY{k}{assert} \PY{p}{(}\PY{n}{M\PYZus{}test\PYZus{}reduced}\PY{o}{.}\PY{n}{shape}\PY{p}{[}\PY{l+m+mi}{0}\PY{p}{]} \PY{o}{==} \PY{l+m+mi}{10}\PY{p}{)}\PY{p}{,} \PY{l+s+s2}{\PYZdq{}}\PY{l+s+s2}{M\PYZus{}reduced has }\PY{l+s+si}{\PYZob{}\PYZcb{}}\PY{l+s+s2}{ rows; should have }\PY{l+s+si}{\PYZob{}\PYZcb{}}\PY{l+s+s2}{\PYZdq{}}\PY{o}{.}\PY{n}{format}\PY{p}{(}\PY{n}{M\PYZus{}test\PYZus{}reduced}\PY{o}{.}\PY{n}{shape}\PY{p}{[}\PY{l+m+mi}{0}\PY{p}{]}\PY{p}{,} \PY{l+m+mi}{10}\PY{p}{)}
        \PY{k}{assert} \PY{p}{(}\PY{n}{M\PYZus{}test\PYZus{}reduced}\PY{o}{.}\PY{n}{shape}\PY{p}{[}\PY{l+m+mi}{1}\PY{p}{]} \PY{o}{==} \PY{l+m+mi}{2}\PY{p}{)}\PY{p}{,} \PY{l+s+s2}{\PYZdq{}}\PY{l+s+s2}{M\PYZus{}reduced has }\PY{l+s+si}{\PYZob{}\PYZcb{}}\PY{l+s+s2}{ columns; should have }\PY{l+s+si}{\PYZob{}\PYZcb{}}\PY{l+s+s2}{\PYZdq{}}\PY{o}{.}\PY{n}{format}\PY{p}{(}\PY{n}{M\PYZus{}test\PYZus{}reduced}\PY{o}{.}\PY{n}{shape}\PY{p}{[}\PY{l+m+mi}{1}\PY{p}{]}\PY{p}{,} \PY{l+m+mi}{2}\PY{p}{)}
        
        \PY{c+c1}{\PYZsh{} Print Success}
        \PY{n+nb}{print} \PY{p}{(}\PY{l+s+s2}{\PYZdq{}}\PY{l+s+s2}{\PYZhy{}}\PY{l+s+s2}{\PYZdq{}} \PY{o}{*} \PY{l+m+mi}{80}\PY{p}{)}
        \PY{n+nb}{print}\PY{p}{(}\PY{l+s+s2}{\PYZdq{}}\PY{l+s+s2}{Passed All Tests!}\PY{l+s+s2}{\PYZdq{}}\PY{p}{)}
        \PY{n+nb}{print} \PY{p}{(}\PY{l+s+s2}{\PYZdq{}}\PY{l+s+s2}{\PYZhy{}}\PY{l+s+s2}{\PYZdq{}} \PY{o}{*} \PY{l+m+mi}{80}\PY{p}{)}
\end{Verbatim}

    \begin{Verbatim}[commandchars=\\\{\}]
Running Truncated SVD over 10 words{\ldots}
Done.
--------------------------------------------------------------------------------
Passed All Tests!
--------------------------------------------------------------------------------

    \end{Verbatim}

    \hypertarget{question-1.4-implement-plot_embeddings-code-1-point}{%
\subsubsection{\texorpdfstring{Question 1.4: Implement
\texttt{plot\_embeddings} {[}code{]} (1
point)}{Question 1.4: Implement plot\_embeddings {[}code{]} (1 point)}}\label{question-1.4-implement-plot_embeddings-code-1-point}}

Here you will write a function to plot a set of 2D vectors in 2D space.
For graphs, we will use Matplotlib (\texttt{plt}).

For this example, you may find it useful to adapt
\href{https://www.pythonmembers.club/2018/05/08/matplotlib-scatter-plot-annotate-set-text-at-label-each-point/}{this
code}. In the future, a good way to make a plot is to look at
\href{https://matplotlib.org/gallery/index.html}{the Matplotlib
gallery}, find a plot that looks somewhat like what you want, and adapt
the code they give.

    \begin{Verbatim}[commandchars=\\\{\}]
{\color{incolor}In [{\color{incolor}10}]:} \PY{k}{def} \PY{n+nf}{plot\PYZus{}embeddings}\PY{p}{(}\PY{n}{M\PYZus{}reduced}\PY{p}{,} \PY{n}{word2Ind}\PY{p}{,} \PY{n}{words}\PY{p}{)}\PY{p}{:}
             \PY{l+s+sd}{\PYZdq{}\PYZdq{}\PYZdq{} Plot in a scatterplot the embeddings of the words specified in the list \PYZdq{}words\PYZdq{}.}
         \PY{l+s+sd}{        NOTE: do not plot all the words listed in M\PYZus{}reduced / word2Ind.}
         \PY{l+s+sd}{        Include a label next to each point.}
         \PY{l+s+sd}{        }
         \PY{l+s+sd}{        Params:}
         \PY{l+s+sd}{            M\PYZus{}reduced (numpy matrix of shape (number of unique words in the corpus , 2)): matrix of 2\PYZhy{}dimensioal word embeddings}
         \PY{l+s+sd}{            word2Ind (dict): dictionary that maps word to indices for matrix M}
         \PY{l+s+sd}{            words (list of strings): words whose embeddings we want to visualize}
         \PY{l+s+sd}{    \PYZdq{}\PYZdq{}\PYZdq{}}
         
             \PY{c+c1}{\PYZsh{} \PYZhy{}\PYZhy{}\PYZhy{}\PYZhy{}\PYZhy{}\PYZhy{}\PYZhy{}\PYZhy{}\PYZhy{}\PYZhy{}\PYZhy{}\PYZhy{}\PYZhy{}\PYZhy{}\PYZhy{}\PYZhy{}\PYZhy{}\PYZhy{}}
             \PY{c+c1}{\PYZsh{} Write your implementation here.}
             
             \PY{n}{x\PYZus{}coords} \PY{o}{=} \PY{n}{M\PYZus{}reduced}\PY{p}{[}\PY{p}{:}\PY{p}{,} \PY{l+m+mi}{0}\PY{p}{]}
             \PY{n}{y\PYZus{}coords} \PY{o}{=} \PY{n}{M\PYZus{}reduced}\PY{p}{[}\PY{p}{:}\PY{p}{,} \PY{l+m+mi}{1}\PY{p}{]}
             
             \PY{k}{for} \PY{n}{word} \PY{o+ow}{in} \PY{n}{words}\PY{p}{:}
                 \PY{n}{idx} \PY{o}{=} \PY{n}{word2Ind}\PY{p}{[}\PY{n}{word}\PY{p}{]}
                 \PY{n}{embedding} \PY{o}{=} \PY{n}{M\PYZus{}reduced}\PY{p}{[}\PY{n}{idx}\PY{p}{]}
                 \PY{n}{x} \PY{o}{=} \PY{n}{embedding}\PY{p}{[}\PY{l+m+mi}{0}\PY{p}{]}
                 \PY{n}{y} \PY{o}{=} \PY{n}{embedding}\PY{p}{[}\PY{l+m+mi}{1}\PY{p}{]}
                 
                 \PY{n}{plt}\PY{o}{.}\PY{n}{scatter}\PY{p}{(}\PY{n}{x}\PY{p}{,} \PY{n}{y}\PY{p}{,} \PY{n}{marker}\PY{o}{=}\PY{l+s+s1}{\PYZsq{}}\PY{l+s+s1}{x}\PY{l+s+s1}{\PYZsq{}}\PY{p}{,} \PY{n}{color}\PY{o}{=}\PY{l+s+s1}{\PYZsq{}}\PY{l+s+s1}{red}\PY{l+s+s1}{\PYZsq{}}\PY{p}{)}
                 \PY{n}{plt}\PY{o}{.}\PY{n}{text}\PY{p}{(}\PY{n}{x}\PY{p}{,} \PY{n}{y}\PY{p}{,} \PY{n}{word}\PY{p}{,} \PY{n}{fontsize}\PY{o}{=}\PY{l+m+mi}{9}\PY{p}{)}
         
             \PY{c+c1}{\PYZsh{} \PYZhy{}\PYZhy{}\PYZhy{}\PYZhy{}\PYZhy{}\PYZhy{}\PYZhy{}\PYZhy{}\PYZhy{}\PYZhy{}\PYZhy{}\PYZhy{}\PYZhy{}\PYZhy{}\PYZhy{}\PYZhy{}\PYZhy{}\PYZhy{}}
\end{Verbatim}

    \begin{Verbatim}[commandchars=\\\{\}]
{\color{incolor}In [{\color{incolor}11}]:} \PY{c+c1}{\PYZsh{} \PYZhy{}\PYZhy{}\PYZhy{}\PYZhy{}\PYZhy{}\PYZhy{}\PYZhy{}\PYZhy{}\PYZhy{}\PYZhy{}\PYZhy{}\PYZhy{}\PYZhy{}\PYZhy{}\PYZhy{}\PYZhy{}\PYZhy{}\PYZhy{}\PYZhy{}\PYZhy{}\PYZhy{}}
         \PY{c+c1}{\PYZsh{} Run this sanity check}
         \PY{c+c1}{\PYZsh{} Note that this is not an exhaustive check for correctness.}
         \PY{c+c1}{\PYZsh{} The plot produced should look like the \PYZdq{}test solution plot\PYZdq{} depicted below. }
         \PY{c+c1}{\PYZsh{} \PYZhy{}\PYZhy{}\PYZhy{}\PYZhy{}\PYZhy{}\PYZhy{}\PYZhy{}\PYZhy{}\PYZhy{}\PYZhy{}\PYZhy{}\PYZhy{}\PYZhy{}\PYZhy{}\PYZhy{}\PYZhy{}\PYZhy{}\PYZhy{}\PYZhy{}\PYZhy{}\PYZhy{}}
         
         \PY{n+nb}{print} \PY{p}{(}\PY{l+s+s2}{\PYZdq{}}\PY{l+s+s2}{\PYZhy{}}\PY{l+s+s2}{\PYZdq{}} \PY{o}{*} \PY{l+m+mi}{80}\PY{p}{)}
         \PY{n+nb}{print} \PY{p}{(}\PY{l+s+s2}{\PYZdq{}}\PY{l+s+s2}{Outputted Plot:}\PY{l+s+s2}{\PYZdq{}}\PY{p}{)}
         
         \PY{n}{M\PYZus{}reduced\PYZus{}plot\PYZus{}test} \PY{o}{=} \PY{n}{np}\PY{o}{.}\PY{n}{array}\PY{p}{(}\PY{p}{[}\PY{p}{[}\PY{l+m+mi}{1}\PY{p}{,} \PY{l+m+mi}{1}\PY{p}{]}\PY{p}{,} \PY{p}{[}\PY{o}{\PYZhy{}}\PY{l+m+mi}{1}\PY{p}{,} \PY{o}{\PYZhy{}}\PY{l+m+mi}{1}\PY{p}{]}\PY{p}{,} \PY{p}{[}\PY{l+m+mi}{1}\PY{p}{,} \PY{o}{\PYZhy{}}\PY{l+m+mi}{1}\PY{p}{]}\PY{p}{,} \PY{p}{[}\PY{o}{\PYZhy{}}\PY{l+m+mi}{1}\PY{p}{,} \PY{l+m+mi}{1}\PY{p}{]}\PY{p}{,} \PY{p}{[}\PY{l+m+mi}{0}\PY{p}{,} \PY{l+m+mi}{0}\PY{p}{]}\PY{p}{]}\PY{p}{)}
         \PY{n}{word2Ind\PYZus{}plot\PYZus{}test} \PY{o}{=} \PY{p}{\PYZob{}}\PY{l+s+s1}{\PYZsq{}}\PY{l+s+s1}{test1}\PY{l+s+s1}{\PYZsq{}}\PY{p}{:} \PY{l+m+mi}{0}\PY{p}{,} \PY{l+s+s1}{\PYZsq{}}\PY{l+s+s1}{test2}\PY{l+s+s1}{\PYZsq{}}\PY{p}{:} \PY{l+m+mi}{1}\PY{p}{,} \PY{l+s+s1}{\PYZsq{}}\PY{l+s+s1}{test3}\PY{l+s+s1}{\PYZsq{}}\PY{p}{:} \PY{l+m+mi}{2}\PY{p}{,} \PY{l+s+s1}{\PYZsq{}}\PY{l+s+s1}{test4}\PY{l+s+s1}{\PYZsq{}}\PY{p}{:} \PY{l+m+mi}{3}\PY{p}{,} \PY{l+s+s1}{\PYZsq{}}\PY{l+s+s1}{test5}\PY{l+s+s1}{\PYZsq{}}\PY{p}{:} \PY{l+m+mi}{4}\PY{p}{\PYZcb{}}
         \PY{n}{words} \PY{o}{=} \PY{p}{[}\PY{l+s+s1}{\PYZsq{}}\PY{l+s+s1}{test1}\PY{l+s+s1}{\PYZsq{}}\PY{p}{,} \PY{l+s+s1}{\PYZsq{}}\PY{l+s+s1}{test2}\PY{l+s+s1}{\PYZsq{}}\PY{p}{,} \PY{l+s+s1}{\PYZsq{}}\PY{l+s+s1}{test3}\PY{l+s+s1}{\PYZsq{}}\PY{p}{,} \PY{l+s+s1}{\PYZsq{}}\PY{l+s+s1}{test4}\PY{l+s+s1}{\PYZsq{}}\PY{p}{,} \PY{l+s+s1}{\PYZsq{}}\PY{l+s+s1}{test5}\PY{l+s+s1}{\PYZsq{}}\PY{p}{]}
         \PY{n}{plot\PYZus{}embeddings}\PY{p}{(}\PY{n}{M\PYZus{}reduced\PYZus{}plot\PYZus{}test}\PY{p}{,} \PY{n}{word2Ind\PYZus{}plot\PYZus{}test}\PY{p}{,} \PY{n}{words}\PY{p}{)}
         
         \PY{n+nb}{print} \PY{p}{(}\PY{l+s+s2}{\PYZdq{}}\PY{l+s+s2}{\PYZhy{}}\PY{l+s+s2}{\PYZdq{}} \PY{o}{*} \PY{l+m+mi}{80}\PY{p}{)}
\end{Verbatim}

    \begin{Verbatim}[commandchars=\\\{\}]
--------------------------------------------------------------------------------
Outputted Plot:
--------------------------------------------------------------------------------

    \end{Verbatim}

    \begin{center}
    \adjustimage{max size={0.9\linewidth}{0.9\paperheight}}{output_20_1.png}
    \end{center}
    { \hspace*{\fill} \\}
    
    \textbf{Test Plot Solution} 

    \hypertarget{question-1.5-co-occurrence-plot-analysis-written-3-points}{%
\subsubsection{Question 1.5: Co-Occurrence Plot Analysis {[}written{]}
(3
points)}\label{question-1.5-co-occurrence-plot-analysis-written-3-points}}

Now we will put together all the parts you have written! We will compute
the co-occurrence matrix with fixed window of 4 (the default window
size), over the Reuters ``crude'' (oil) corpus. Then we will use
TruncatedSVD to compute 2-dimensional embeddings of each word.
TruncatedSVD returns U*S, so we need to normalize the returned vectors,
so that all the vectors will appear around the unit circle (therefore
closeness is directional closeness). \textbf{Note}: The line of code
below that does the normalizing uses the NumPy concept of
\emph{broadcasting}. If you don't know about broadcasting, check out
\href{https://jakevdp.github.io/PythonDataScienceHandbook/02.05-computation-on-arrays-broadcasting.html}{Computation
on Arrays: Broadcasting by Jake VanderPlas}.

Run the below cell to produce the plot. It'll probably take a few
seconds to run. What clusters together in 2-dimensional embedding space?
What doesn't cluster together that you might think should have?
\textbf{Note:} ``bpd'' stands for ``barrels per day'' and is a commonly
used abbreviation in crude oil topic articles.

    \begin{Verbatim}[commandchars=\\\{\}]
{\color{incolor}In [{\color{incolor}12}]:} \PY{c+c1}{\PYZsh{} \PYZhy{}\PYZhy{}\PYZhy{}\PYZhy{}\PYZhy{}\PYZhy{}\PYZhy{}\PYZhy{}\PYZhy{}\PYZhy{}\PYZhy{}\PYZhy{}\PYZhy{}\PYZhy{}\PYZhy{}\PYZhy{}\PYZhy{}\PYZhy{}\PYZhy{}\PYZhy{}\PYZhy{}\PYZhy{}\PYZhy{}\PYZhy{}\PYZhy{}\PYZhy{}\PYZhy{}\PYZhy{}\PYZhy{}}
         \PY{c+c1}{\PYZsh{} Run This Cell to Produce Your Plot}
         \PY{c+c1}{\PYZsh{} \PYZhy{}\PYZhy{}\PYZhy{}\PYZhy{}\PYZhy{}\PYZhy{}\PYZhy{}\PYZhy{}\PYZhy{}\PYZhy{}\PYZhy{}\PYZhy{}\PYZhy{}\PYZhy{}\PYZhy{}\PYZhy{}\PYZhy{}\PYZhy{}\PYZhy{}\PYZhy{}\PYZhy{}\PYZhy{}\PYZhy{}\PYZhy{}\PYZhy{}\PYZhy{}\PYZhy{}\PYZhy{}\PYZhy{}\PYZhy{}}
         \PY{n}{reuters\PYZus{}corpus} \PY{o}{=} \PY{n}{read\PYZus{}corpus}\PY{p}{(}\PY{p}{)}
         \PY{n}{M\PYZus{}co\PYZus{}occurrence}\PY{p}{,} \PY{n}{word2Ind\PYZus{}co\PYZus{}occurrence} \PY{o}{=} \PY{n}{compute\PYZus{}co\PYZus{}occurrence\PYZus{}matrix}\PY{p}{(}\PY{n}{reuters\PYZus{}corpus}\PY{p}{)}
         \PY{n}{M\PYZus{}reduced\PYZus{}co\PYZus{}occurrence} \PY{o}{=} \PY{n}{reduce\PYZus{}to\PYZus{}k\PYZus{}dim}\PY{p}{(}\PY{n}{M\PYZus{}co\PYZus{}occurrence}\PY{p}{,} \PY{n}{k}\PY{o}{=}\PY{l+m+mi}{2}\PY{p}{)}
         
         \PY{c+c1}{\PYZsh{} Rescale (normalize) the rows to make them each of unit\PYZhy{}length}
         \PY{n}{M\PYZus{}lengths} \PY{o}{=} \PY{n}{np}\PY{o}{.}\PY{n}{linalg}\PY{o}{.}\PY{n}{norm}\PY{p}{(}\PY{n}{M\PYZus{}reduced\PYZus{}co\PYZus{}occurrence}\PY{p}{,} \PY{n}{axis}\PY{o}{=}\PY{l+m+mi}{1}\PY{p}{)}
         \PY{n}{M\PYZus{}normalized} \PY{o}{=} \PY{n}{M\PYZus{}reduced\PYZus{}co\PYZus{}occurrence} \PY{o}{/} \PY{n}{M\PYZus{}lengths}\PY{p}{[}\PY{p}{:}\PY{p}{,} \PY{n}{np}\PY{o}{.}\PY{n}{newaxis}\PY{p}{]} \PY{c+c1}{\PYZsh{} broadcasting}
         
         \PY{n}{words} \PY{o}{=} \PY{p}{[}\PY{l+s+s1}{\PYZsq{}}\PY{l+s+s1}{barrels}\PY{l+s+s1}{\PYZsq{}}\PY{p}{,} \PY{l+s+s1}{\PYZsq{}}\PY{l+s+s1}{bpd}\PY{l+s+s1}{\PYZsq{}}\PY{p}{,} \PY{l+s+s1}{\PYZsq{}}\PY{l+s+s1}{ecuador}\PY{l+s+s1}{\PYZsq{}}\PY{p}{,} \PY{l+s+s1}{\PYZsq{}}\PY{l+s+s1}{energy}\PY{l+s+s1}{\PYZsq{}}\PY{p}{,} \PY{l+s+s1}{\PYZsq{}}\PY{l+s+s1}{industry}\PY{l+s+s1}{\PYZsq{}}\PY{p}{,} \PY{l+s+s1}{\PYZsq{}}\PY{l+s+s1}{kuwait}\PY{l+s+s1}{\PYZsq{}}\PY{p}{,} \PY{l+s+s1}{\PYZsq{}}\PY{l+s+s1}{oil}\PY{l+s+s1}{\PYZsq{}}\PY{p}{,} \PY{l+s+s1}{\PYZsq{}}\PY{l+s+s1}{output}\PY{l+s+s1}{\PYZsq{}}\PY{p}{,} \PY{l+s+s1}{\PYZsq{}}\PY{l+s+s1}{petroleum}\PY{l+s+s1}{\PYZsq{}}\PY{p}{,} \PY{l+s+s1}{\PYZsq{}}\PY{l+s+s1}{venezuela}\PY{l+s+s1}{\PYZsq{}}\PY{p}{]}
         
         \PY{n}{plot\PYZus{}embeddings}\PY{p}{(}\PY{n}{M\PYZus{}normalized}\PY{p}{,} \PY{n}{word2Ind\PYZus{}co\PYZus{}occurrence}\PY{p}{,} \PY{n}{words}\PY{p}{)}
\end{Verbatim}

    \begin{Verbatim}[commandchars=\\\{\}]
Running Truncated SVD over 8185 words{\ldots}
Done.

    \end{Verbatim}

    \begin{center}
    \adjustimage{max size={0.9\linewidth}{0.9\paperheight}}{output_23_1.png}
    \end{center}
    { \hspace*{\fill} \\}
    
    \hypertarget{write-your-answer-here.}{%
\paragraph{Write your answer here.}\label{write-your-answer-here.}}

What clusters together in 2-dimensional embedding space?

Cluster one (countries): Kuwait, Venezuela, Ecuador. Cluster two
(resources): oil, energy. Culster three (function): petroleum, industry.
Cluster four (production): barrels, output.

What doesn't cluster together that you might think should have? BPD did
not cluster with barrels and output, possibly because it is an acronym.
Output is not inclusive to just barrel production, but those three terms
could be a hypothetical cluster.

    \hypertarget{part-2-prediction-based-word-vectors-15-points}{%
\subsection{Part 2: Prediction-Based Word Vectors (15
points)}\label{part-2-prediction-based-word-vectors-15-points}}

As discussed in class, more recently prediction-based word vectors have
demonstrated better performance, such as word2vec and GloVe (which also
utilizes the benefit of counts). Here, we shall explore the embeddings
produced by GloVe. Please revisit the class notes and lecture slides for
more details on the word2vec and GloVe algorithms. If you're feeling
adventurous, challenge yourself and try reading
\href{https://nlp.stanford.edu/pubs/glove.pdf}{GloVe's original paper}.

Then run the following cells to load the GloVe vectors into memory.
\textbf{Note}: If this is your first time to run these cells,
i.e.~download the embedding model, it will take about 15 minutes to run.
If you've run these cells before, rerunning them will load the model
without redownloading it, which will take about 1 to 2 minutes.

    \begin{Verbatim}[commandchars=\\\{\}]
{\color{incolor}In [{\color{incolor}13}]:} \PY{k}{def} \PY{n+nf}{load\PYZus{}embedding\PYZus{}model}\PY{p}{(}\PY{p}{)}\PY{p}{:}
             \PY{l+s+sd}{\PYZdq{}\PYZdq{}\PYZdq{} Load GloVe Vectors}
         \PY{l+s+sd}{        Return:}
         \PY{l+s+sd}{            wv\PYZus{}from\PYZus{}bin: All 400000 embeddings, each lengh 200}
         \PY{l+s+sd}{    \PYZdq{}\PYZdq{}\PYZdq{}}
             \PY{k+kn}{import} \PY{n+nn}{gensim}\PY{n+nn}{.}\PY{n+nn}{downloader} \PY{k}{as} \PY{n+nn}{api}
             \PY{n}{wv\PYZus{}from\PYZus{}bin} \PY{o}{=} \PY{n}{api}\PY{o}{.}\PY{n}{load}\PY{p}{(}\PY{l+s+s2}{\PYZdq{}}\PY{l+s+s2}{glove\PYZhy{}wiki\PYZhy{}gigaword\PYZhy{}200}\PY{l+s+s2}{\PYZdq{}}\PY{p}{)}
             \PY{n+nb}{print}\PY{p}{(}\PY{l+s+s2}{\PYZdq{}}\PY{l+s+s2}{Loaded vocab size }\PY{l+s+si}{\PYZpc{}i}\PY{l+s+s2}{\PYZdq{}} \PY{o}{\PYZpc{}} \PY{n+nb}{len}\PY{p}{(}\PY{n}{wv\PYZus{}from\PYZus{}bin}\PY{o}{.}\PY{n}{vocab}\PY{o}{.}\PY{n}{keys}\PY{p}{(}\PY{p}{)}\PY{p}{)}\PY{p}{)}
             \PY{k}{return} \PY{n}{wv\PYZus{}from\PYZus{}bin}
\end{Verbatim}

    \begin{Verbatim}[commandchars=\\\{\}]
{\color{incolor}In [{\color{incolor}14}]:} \PY{c+c1}{\PYZsh{} \PYZhy{}\PYZhy{}\PYZhy{}\PYZhy{}\PYZhy{}\PYZhy{}\PYZhy{}\PYZhy{}\PYZhy{}\PYZhy{}\PYZhy{}\PYZhy{}\PYZhy{}\PYZhy{}\PYZhy{}\PYZhy{}\PYZhy{}\PYZhy{}\PYZhy{}\PYZhy{}\PYZhy{}\PYZhy{}\PYZhy{}\PYZhy{}\PYZhy{}\PYZhy{}\PYZhy{}\PYZhy{}\PYZhy{}\PYZhy{}\PYZhy{}\PYZhy{}\PYZhy{}\PYZhy{}\PYZhy{}}
         \PY{c+c1}{\PYZsh{} Run Cell to Load Word Vectors}
         \PY{c+c1}{\PYZsh{} Note: This will take several minutes}
         \PY{c+c1}{\PYZsh{} \PYZhy{}\PYZhy{}\PYZhy{}\PYZhy{}\PYZhy{}\PYZhy{}\PYZhy{}\PYZhy{}\PYZhy{}\PYZhy{}\PYZhy{}\PYZhy{}\PYZhy{}\PYZhy{}\PYZhy{}\PYZhy{}\PYZhy{}\PYZhy{}\PYZhy{}\PYZhy{}\PYZhy{}\PYZhy{}\PYZhy{}\PYZhy{}\PYZhy{}\PYZhy{}\PYZhy{}\PYZhy{}\PYZhy{}\PYZhy{}\PYZhy{}\PYZhy{}\PYZhy{}\PYZhy{}\PYZhy{}}
         \PY{n}{wv\PYZus{}from\PYZus{}bin} \PY{o}{=} \PY{n}{load\PYZus{}embedding\PYZus{}model}\PY{p}{(}\PY{p}{)}
\end{Verbatim}

    \begin{Verbatim}[commandchars=\\\{\}]
Loaded vocab size 400000

    \end{Verbatim}

    \hypertarget{note-if-you-are-receiving-reset-by-peer-error-rerun-the-cell-to-restart-the-download.}{%
\paragraph{Note: If you are receiving reset by peer error, rerun the
cell to restart the
download.}\label{note-if-you-are-receiving-reset-by-peer-error-rerun-the-cell-to-restart-the-download.}}

    \hypertarget{reducing-dimensionality-of-word-embeddings}{%
\subsubsection{Reducing dimensionality of Word
Embeddings}\label{reducing-dimensionality-of-word-embeddings}}

Let's directly compare the GloVe embeddings to those of the
co-occurrence matrix. In order to avoid running out of memory, we will
work with a sample of 10000 GloVe vectors instead. Run the following
cells to:

\begin{enumerate}
\def\labelenumi{\arabic{enumi}.}
\tightlist
\item
  Put 10000 Glove vectors into a matrix M
\item
  Run reduce\_to\_k\_dim (your Truncated SVD function) to reduce the
  vectors from 200-dimensional to 2-dimensional.
\end{enumerate}

    \begin{Verbatim}[commandchars=\\\{\}]
{\color{incolor}In [{\color{incolor}15}]:} \PY{k}{def} \PY{n+nf}{get\PYZus{}matrix\PYZus{}of\PYZus{}vectors}\PY{p}{(}\PY{n}{wv\PYZus{}from\PYZus{}bin}\PY{p}{,} \PY{n}{required\PYZus{}words}\PY{o}{=}\PY{p}{[}\PY{l+s+s1}{\PYZsq{}}\PY{l+s+s1}{barrels}\PY{l+s+s1}{\PYZsq{}}\PY{p}{,} \PY{l+s+s1}{\PYZsq{}}\PY{l+s+s1}{bpd}\PY{l+s+s1}{\PYZsq{}}\PY{p}{,} \PY{l+s+s1}{\PYZsq{}}\PY{l+s+s1}{ecuador}\PY{l+s+s1}{\PYZsq{}}\PY{p}{,} \PY{l+s+s1}{\PYZsq{}}\PY{l+s+s1}{energy}\PY{l+s+s1}{\PYZsq{}}\PY{p}{,} \PY{l+s+s1}{\PYZsq{}}\PY{l+s+s1}{industry}\PY{l+s+s1}{\PYZsq{}}\PY{p}{,} \PY{l+s+s1}{\PYZsq{}}\PY{l+s+s1}{kuwait}\PY{l+s+s1}{\PYZsq{}}\PY{p}{,} \PY{l+s+s1}{\PYZsq{}}\PY{l+s+s1}{oil}\PY{l+s+s1}{\PYZsq{}}\PY{p}{,} \PY{l+s+s1}{\PYZsq{}}\PY{l+s+s1}{output}\PY{l+s+s1}{\PYZsq{}}\PY{p}{,} \PY{l+s+s1}{\PYZsq{}}\PY{l+s+s1}{petroleum}\PY{l+s+s1}{\PYZsq{}}\PY{p}{,} \PY{l+s+s1}{\PYZsq{}}\PY{l+s+s1}{venezuela}\PY{l+s+s1}{\PYZsq{}}\PY{p}{]}\PY{p}{)}\PY{p}{:}
             \PY{l+s+sd}{\PYZdq{}\PYZdq{}\PYZdq{} Put the GloVe vectors into a matrix M.}
         \PY{l+s+sd}{        Param:}
         \PY{l+s+sd}{            wv\PYZus{}from\PYZus{}bin: KeyedVectors object; the 400000 GloVe vectors loaded from file}
         \PY{l+s+sd}{        Return:}
         \PY{l+s+sd}{            M: numpy matrix shape (num words, 200) containing the vectors}
         \PY{l+s+sd}{            word2Ind: dictionary mapping each word to its row number in M}
         \PY{l+s+sd}{    \PYZdq{}\PYZdq{}\PYZdq{}}
             \PY{k+kn}{import} \PY{n+nn}{random}
             \PY{n}{words} \PY{o}{=} \PY{n+nb}{list}\PY{p}{(}\PY{n}{wv\PYZus{}from\PYZus{}bin}\PY{o}{.}\PY{n}{vocab}\PY{o}{.}\PY{n}{keys}\PY{p}{(}\PY{p}{)}\PY{p}{)}
             \PY{n+nb}{print}\PY{p}{(}\PY{l+s+s2}{\PYZdq{}}\PY{l+s+s2}{Shuffling words ...}\PY{l+s+s2}{\PYZdq{}}\PY{p}{)}
             \PY{n}{random}\PY{o}{.}\PY{n}{seed}\PY{p}{(}\PY{l+m+mi}{224}\PY{p}{)}
             \PY{n}{random}\PY{o}{.}\PY{n}{shuffle}\PY{p}{(}\PY{n}{words}\PY{p}{)}
             \PY{n}{words} \PY{o}{=} \PY{n}{words}\PY{p}{[}\PY{p}{:}\PY{l+m+mi}{10000}\PY{p}{]}
             \PY{n+nb}{print}\PY{p}{(}\PY{l+s+s2}{\PYZdq{}}\PY{l+s+s2}{Putting }\PY{l+s+si}{\PYZpc{}i}\PY{l+s+s2}{ words into word2Ind and matrix M...}\PY{l+s+s2}{\PYZdq{}} \PY{o}{\PYZpc{}} \PY{n+nb}{len}\PY{p}{(}\PY{n}{words}\PY{p}{)}\PY{p}{)}
             \PY{n}{word2Ind} \PY{o}{=} \PY{p}{\PYZob{}}\PY{p}{\PYZcb{}}
             \PY{n}{M} \PY{o}{=} \PY{p}{[}\PY{p}{]}
             \PY{n}{curInd} \PY{o}{=} \PY{l+m+mi}{0}
             \PY{k}{for} \PY{n}{w} \PY{o+ow}{in} \PY{n}{words}\PY{p}{:}
                 \PY{k}{try}\PY{p}{:}
                     \PY{n}{M}\PY{o}{.}\PY{n}{append}\PY{p}{(}\PY{n}{wv\PYZus{}from\PYZus{}bin}\PY{o}{.}\PY{n}{word\PYZus{}vec}\PY{p}{(}\PY{n}{w}\PY{p}{)}\PY{p}{)}
                     \PY{n}{word2Ind}\PY{p}{[}\PY{n}{w}\PY{p}{]} \PY{o}{=} \PY{n}{curInd}
                     \PY{n}{curInd} \PY{o}{+}\PY{o}{=} \PY{l+m+mi}{1}
                 \PY{k}{except} \PY{n+ne}{KeyError}\PY{p}{:}
                     \PY{k}{continue}
             \PY{k}{for} \PY{n}{w} \PY{o+ow}{in} \PY{n}{required\PYZus{}words}\PY{p}{:}
                 \PY{k}{if} \PY{n}{w} \PY{o+ow}{in} \PY{n}{words}\PY{p}{:}
                     \PY{k}{continue}
                 \PY{k}{try}\PY{p}{:}
                     \PY{n}{M}\PY{o}{.}\PY{n}{append}\PY{p}{(}\PY{n}{wv\PYZus{}from\PYZus{}bin}\PY{o}{.}\PY{n}{word\PYZus{}vec}\PY{p}{(}\PY{n}{w}\PY{p}{)}\PY{p}{)}
                     \PY{n}{word2Ind}\PY{p}{[}\PY{n}{w}\PY{p}{]} \PY{o}{=} \PY{n}{curInd}
                     \PY{n}{curInd} \PY{o}{+}\PY{o}{=} \PY{l+m+mi}{1}
                 \PY{k}{except} \PY{n+ne}{KeyError}\PY{p}{:}
                     \PY{k}{continue}
             \PY{n}{M} \PY{o}{=} \PY{n}{np}\PY{o}{.}\PY{n}{stack}\PY{p}{(}\PY{n}{M}\PY{p}{)}
             \PY{n+nb}{print}\PY{p}{(}\PY{l+s+s2}{\PYZdq{}}\PY{l+s+s2}{Done.}\PY{l+s+s2}{\PYZdq{}}\PY{p}{)}
             \PY{k}{return} \PY{n}{M}\PY{p}{,} \PY{n}{word2Ind}
\end{Verbatim}

    \begin{Verbatim}[commandchars=\\\{\}]
{\color{incolor}In [{\color{incolor}16}]:} \PY{c+c1}{\PYZsh{} \PYZhy{}\PYZhy{}\PYZhy{}\PYZhy{}\PYZhy{}\PYZhy{}\PYZhy{}\PYZhy{}\PYZhy{}\PYZhy{}\PYZhy{}\PYZhy{}\PYZhy{}\PYZhy{}\PYZhy{}\PYZhy{}\PYZhy{}\PYZhy{}\PYZhy{}\PYZhy{}\PYZhy{}\PYZhy{}\PYZhy{}\PYZhy{}\PYZhy{}\PYZhy{}\PYZhy{}\PYZhy{}\PYZhy{}\PYZhy{}\PYZhy{}\PYZhy{}\PYZhy{}\PYZhy{}\PYZhy{}\PYZhy{}\PYZhy{}\PYZhy{}\PYZhy{}\PYZhy{}\PYZhy{}\PYZhy{}\PYZhy{}\PYZhy{}\PYZhy{}\PYZhy{}\PYZhy{}\PYZhy{}\PYZhy{}\PYZhy{}\PYZhy{}\PYZhy{}\PYZhy{}\PYZhy{}\PYZhy{}\PYZhy{}\PYZhy{}\PYZhy{}\PYZhy{}\PYZhy{}\PYZhy{}\PYZhy{}\PYZhy{}\PYZhy{}\PYZhy{}}
         \PY{c+c1}{\PYZsh{} Run Cell to Reduce 200\PYZhy{}Dimensional Word Embeddings to k Dimensions}
         \PY{c+c1}{\PYZsh{} Note: This should be quick to run}
         \PY{c+c1}{\PYZsh{} \PYZhy{}\PYZhy{}\PYZhy{}\PYZhy{}\PYZhy{}\PYZhy{}\PYZhy{}\PYZhy{}\PYZhy{}\PYZhy{}\PYZhy{}\PYZhy{}\PYZhy{}\PYZhy{}\PYZhy{}\PYZhy{}\PYZhy{}\PYZhy{}\PYZhy{}\PYZhy{}\PYZhy{}\PYZhy{}\PYZhy{}\PYZhy{}\PYZhy{}\PYZhy{}\PYZhy{}\PYZhy{}\PYZhy{}\PYZhy{}\PYZhy{}\PYZhy{}\PYZhy{}\PYZhy{}\PYZhy{}\PYZhy{}\PYZhy{}\PYZhy{}\PYZhy{}\PYZhy{}\PYZhy{}\PYZhy{}\PYZhy{}\PYZhy{}\PYZhy{}\PYZhy{}\PYZhy{}\PYZhy{}\PYZhy{}\PYZhy{}\PYZhy{}\PYZhy{}\PYZhy{}\PYZhy{}\PYZhy{}\PYZhy{}\PYZhy{}\PYZhy{}\PYZhy{}\PYZhy{}\PYZhy{}\PYZhy{}\PYZhy{}\PYZhy{}\PYZhy{}}
         \PY{n}{M}\PY{p}{,} \PY{n}{word2Ind} \PY{o}{=} \PY{n}{get\PYZus{}matrix\PYZus{}of\PYZus{}vectors}\PY{p}{(}\PY{n}{wv\PYZus{}from\PYZus{}bin}\PY{p}{)}
         \PY{n}{M\PYZus{}reduced} \PY{o}{=} \PY{n}{reduce\PYZus{}to\PYZus{}k\PYZus{}dim}\PY{p}{(}\PY{n}{M}\PY{p}{,} \PY{n}{k}\PY{o}{=}\PY{l+m+mi}{2}\PY{p}{)}
         
         \PY{c+c1}{\PYZsh{} Rescale (normalize) the rows to make them each of unit\PYZhy{}length}
         \PY{n}{M\PYZus{}lengths} \PY{o}{=} \PY{n}{np}\PY{o}{.}\PY{n}{linalg}\PY{o}{.}\PY{n}{norm}\PY{p}{(}\PY{n}{M\PYZus{}reduced}\PY{p}{,} \PY{n}{axis}\PY{o}{=}\PY{l+m+mi}{1}\PY{p}{)}
         \PY{n}{M\PYZus{}reduced\PYZus{}normalized} \PY{o}{=} \PY{n}{M\PYZus{}reduced} \PY{o}{/} \PY{n}{M\PYZus{}lengths}\PY{p}{[}\PY{p}{:}\PY{p}{,} \PY{n}{np}\PY{o}{.}\PY{n}{newaxis}\PY{p}{]} \PY{c+c1}{\PYZsh{} broadcasting}
\end{Verbatim}

    \begin{Verbatim}[commandchars=\\\{\}]
Shuffling words {\ldots}
Putting 10000 words into word2Ind and matrix M{\ldots}
Done.
Running Truncated SVD over 10010 words{\ldots}
Done.

    \end{Verbatim}

    \hypertarget{question-2.1-glove-plot-analysis-written-4-points}{%
\subsubsection{Question 2.1: GloVe Plot Analysis {[}written{]} (4
points)}\label{question-2.1-glove-plot-analysis-written-4-points}}

Run the cell below to plot the 2D GloVe embeddings for
\texttt{{[}\textquotesingle{}barrels\textquotesingle{},\ \textquotesingle{}bpd\textquotesingle{},\ \textquotesingle{}ecuador\textquotesingle{},\ \textquotesingle{}energy\textquotesingle{},\ \textquotesingle{}industry\textquotesingle{},\ \textquotesingle{}kuwait\textquotesingle{},\ \textquotesingle{}oil\textquotesingle{},\ \textquotesingle{}output\textquotesingle{},\ \textquotesingle{}petroleum\textquotesingle{},\ \textquotesingle{}venezuela\textquotesingle{}{]}}.

What clusters together in 2-dimensional embedding space? What doesn't
cluster together that you might think should have? How is the plot
different from the one generated earlier from the co-occurrence matrix?
What is a possible reason for causing the difference?

    \begin{Verbatim}[commandchars=\\\{\}]
{\color{incolor}In [{\color{incolor}17}]:} \PY{n}{words} \PY{o}{=} \PY{p}{[}\PY{l+s+s1}{\PYZsq{}}\PY{l+s+s1}{barrels}\PY{l+s+s1}{\PYZsq{}}\PY{p}{,} \PY{l+s+s1}{\PYZsq{}}\PY{l+s+s1}{bpd}\PY{l+s+s1}{\PYZsq{}}\PY{p}{,} \PY{l+s+s1}{\PYZsq{}}\PY{l+s+s1}{ecuador}\PY{l+s+s1}{\PYZsq{}}\PY{p}{,} \PY{l+s+s1}{\PYZsq{}}\PY{l+s+s1}{energy}\PY{l+s+s1}{\PYZsq{}}\PY{p}{,} \PY{l+s+s1}{\PYZsq{}}\PY{l+s+s1}{industry}\PY{l+s+s1}{\PYZsq{}}\PY{p}{,} \PY{l+s+s1}{\PYZsq{}}\PY{l+s+s1}{kuwait}\PY{l+s+s1}{\PYZsq{}}\PY{p}{,} \PY{l+s+s1}{\PYZsq{}}\PY{l+s+s1}{oil}\PY{l+s+s1}{\PYZsq{}}\PY{p}{,} \PY{l+s+s1}{\PYZsq{}}\PY{l+s+s1}{output}\PY{l+s+s1}{\PYZsq{}}\PY{p}{,} \PY{l+s+s1}{\PYZsq{}}\PY{l+s+s1}{petroleum}\PY{l+s+s1}{\PYZsq{}}\PY{p}{,} \PY{l+s+s1}{\PYZsq{}}\PY{l+s+s1}{venezuela}\PY{l+s+s1}{\PYZsq{}}\PY{p}{]}
         \PY{n}{plot\PYZus{}embeddings}\PY{p}{(}\PY{n}{M\PYZus{}reduced\PYZus{}normalized}\PY{p}{,} \PY{n}{word2Ind}\PY{p}{,} \PY{n}{words}\PY{p}{)}
\end{Verbatim}

    \begin{center}
    \adjustimage{max size={0.9\linewidth}{0.9\paperheight}}{output_33_0.png}
    \end{center}
    { \hspace*{\fill} \\}
    
    \hypertarget{write-your-answer-here.}{%
\paragraph{Write your answer here.}\label{write-your-answer-here.}}

What clusters together in 2-dimensional embedding space?

Cluster one (countries): Kuwait, Venezuela, Ecuador, oil, energy,
petroleum, and industry. Cluster two (production): barrels, output.

What doesn't cluster together that you might think should have? BPD did
not cluster with barrels and output, possibly because it is an acronym.
Output is not inclusive to just barrel production, but those three terms
could be a hypothetical cluster.

How is the plot different from the one generated earlier from the
co-occurrence matrix? There is more distance between the outlier words
like bdp, barrels, and output on this plot compared to the smae words
for the first plot.

What is a possible reason for causing the difference? With GloVe, the
loss function is the difference between the product of the vector
differences and taking the log of the probability of co-occurrence
minimizes the space between word associations.

    \hypertarget{cosine-similarity}{%
\subsubsection{Cosine Similarity}\label{cosine-similarity}}

Now that we have word vectors, we need a way to quantify the similarity
between individual words, according to these vectors. One such metric is
cosine-similarity. We will be using this to find words that are
``close'' and ``far'' from one another.

We can think of n-dimensional vectors as points in n-dimensional space.
If we take this perspective
\href{http://mathworld.wolfram.com/L1-Norm.html}{L1} and
\href{http://mathworld.wolfram.com/L2-Norm.html}{L2} Distances help
quantify the amount of space ``we must travel'' to get between these two
points. Another approach is to examine the angle between two vectors.
From trigonometry we know that:

Instead of computing the actual angle, we can leave the similarity in
terms of \(similarity = cos(\Theta)\). Formally the
\href{https://en.wikipedia.org/wiki/Cosine_similarity}{Cosine
Similarity} \(s\) between two vectors \(p\) and \(q\) is defined as:

\[s = \frac{p \cdot q}{||p|| ||q||}, \textrm{ where } s \in [-1, 1] \]

    \hypertarget{question-2.2-words-with-multiple-meanings-2-points-code-written}{%
\subsubsection{Question 2.2: Words with Multiple Meanings (2 points)
{[}code +
written{]}}\label{question-2.2-words-with-multiple-meanings-2-points-code-written}}

Polysemes and homonyms are words that have more than one meaning (see
this \href{https://en.wikipedia.org/wiki/Polysemy}{wiki page} to learn
more about the difference between polysemes and homonyms ). Find a word
with at least 2 different meanings such that the top-10 most similar
words (according to cosine similarity) contain related words from
\emph{both} meanings. For example, ``leaves'' has both ``vanishes'' and
``stalks'' in the top 10, and ``scoop'' has both
``handed\_waffle\_cone'' and ``lowdown''. You will probably need to try
several polysemous or homonymic words before you find one. Please state
the word you discover and the multiple meanings that occur in the top
10. Why do you think many of the polysemous or homonymic words you tried
didn't work (i.e.~the top-10 most similar words only contain
\textbf{one} of the meanings of the words)?

\textbf{Note}: You should use the
\texttt{wv\_from\_bin.most\_similar(word)} function to get the top 10
similar words. This function ranks all other words in the vocabulary
with respect to their cosine similarity to the given word. For further
assistance please check the
\textbf{\href{https://radimrehurek.com/gensim/models/keyedvectors.html\#gensim.models.keyedvectors.FastTextKeyedVectors.most_similar}{GenSim
documentation}}.

    \begin{Verbatim}[commandchars=\\\{\}]
{\color{incolor}In [{\color{incolor}18}]:}     \PY{c+c1}{\PYZsh{} \PYZhy{}\PYZhy{}\PYZhy{}\PYZhy{}\PYZhy{}\PYZhy{}\PYZhy{}\PYZhy{}\PYZhy{}\PYZhy{}\PYZhy{}\PYZhy{}\PYZhy{}\PYZhy{}\PYZhy{}\PYZhy{}\PYZhy{}\PYZhy{}}
             \PY{c+c1}{\PYZsh{} Write your implementation here.}
             \PY{n}{polyseme} \PY{o}{=} \PY{l+s+s2}{\PYZdq{}}\PY{l+s+s2}{procure}\PY{l+s+s2}{\PYZdq{}}
             \PY{n+nb}{print}\PY{p}{(}\PY{l+s+s2}{\PYZdq{}}\PY{l+s+s2}{polyseme}\PY{l+s+s2}{\PYZdq{}}\PY{p}{)}
             \PY{n+nb}{list}\PY{p}{(}\PY{n}{wv\PYZus{}from\PYZus{}bin}\PY{o}{.}\PY{n}{most\PYZus{}similar}\PY{p}{(}\PY{n}{polyseme}\PY{p}{)}\PY{p}{)}
         
             \PY{c+c1}{\PYZsh{} \PYZhy{}\PYZhy{}\PYZhy{}\PYZhy{}\PYZhy{}\PYZhy{}\PYZhy{}\PYZhy{}\PYZhy{}\PYZhy{}\PYZhy{}\PYZhy{}\PYZhy{}\PYZhy{}\PYZhy{}\PYZhy{}\PYZhy{}\PYZhy{}}
\end{Verbatim}

    \begin{Verbatim}[commandchars=\\\{\}]
polyseme

    \end{Verbatim}

\begin{Verbatim}[commandchars=\\\{\}]
{\color{outcolor}Out[{\color{outcolor}18}]:} [('procuring', 0.6463401913642883),
          ('procured', 0.6035971641540527),
          ('obtain', 0.5592821836471558),
          ('equip', 0.5188530087471008),
          ('manufacture', 0.5124149918556213),
          ('fabricate', 0.4991585910320282),
          ('assemble', 0.49078986048698425),
          ('smuggle', 0.48390424251556396),
          ('distribute', 0.4838152527809143),
          ('supplying', 0.48204174637794495)]
\end{Verbatim}
            
    \hypertarget{write-your-answer-here.}{%
\paragraph{Write your answer here.}\label{write-your-answer-here.}}

Please state the word you discover and the multiple meanings that occur
in the top 10. procure, multiple meanings - procuring, procured, obtain,
equip, manufacture, fabricate, assemble, smuggle, distribute, supplying

Why do you think many of the polysemous or homonymic words you tried
didn't work (i.e.~the top-10 most similar words only contain one of the
meanings of the words)? The verb ``procure'' is a good example of
polysemy as it can mean ``become,'' ``get,'' or ``understand''. The
author from the Reuters (business and financial news) corpus uses the
word `procure' in the business or economic sense. The top three words
are closely associated to the general polyseme, rather the definition
within economics.

    \hypertarget{question-2.3-synonyms-antonyms-2-points-code-written}{%
\subsubsection{Question 2.3: Synonyms \& Antonyms (2 points) {[}code +
written{]}}\label{question-2.3-synonyms-antonyms-2-points-code-written}}

When considering Cosine Similarity, it's often more convenient to think
of Cosine Distance, which is simply 1 - Cosine Similarity.

Find three words (w1,w2,w3) where w1 and w2 are synonyms and w1 and w3
are antonyms, but Cosine Distance(w1,w3) \textless{} Cosine
Distance(w1,w2). For example, w1=``happy'' is closer to w3=``sad'' than
to w2=``cheerful''.

Once you have found your example, please give a possible explanation for
why this counter-intuitive result may have happened.

You should use the the \texttt{wv\_from\_bin.distance(w1,\ w2)} function
here in order to compute the cosine distance between two words. Please
see the
\textbf{\href{https://radimrehurek.com/gensim/models/keyedvectors.html\#gensim.models.keyedvectors.FastTextKeyedVectors.distance}{GenSim
documentation}} for further assistance.

    \begin{Verbatim}[commandchars=\\\{\}]
{\color{incolor}In [{\color{incolor}19}]:}     \PY{c+c1}{\PYZsh{} \PYZhy{}\PYZhy{}\PYZhy{}\PYZhy{}\PYZhy{}\PYZhy{}\PYZhy{}\PYZhy{}\PYZhy{}\PYZhy{}\PYZhy{}\PYZhy{}\PYZhy{}\PYZhy{}\PYZhy{}\PYZhy{}\PYZhy{}\PYZhy{}}
             \PY{c+c1}{\PYZsh{} Write your implementation here.}
             \PY{n}{w1} \PY{o}{=} \PY{l+s+s2}{\PYZdq{}}\PY{l+s+s2}{good}\PY{l+s+s2}{\PYZdq{}}
             \PY{n}{w2} \PY{o}{=} \PY{l+s+s2}{\PYZdq{}}\PY{l+s+s2}{fine}\PY{l+s+s2}{\PYZdq{}}
             \PY{n}{w3} \PY{o}{=} \PY{l+s+s2}{\PYZdq{}}\PY{l+s+s2}{bad}\PY{l+s+s2}{\PYZdq{}}
             \PY{n}{w1\PYZus{}w2\PYZus{}dist} \PY{o}{=} \PY{n}{wv\PYZus{}from\PYZus{}bin}\PY{o}{.}\PY{n}{distance}\PY{p}{(}\PY{n}{w1}\PY{p}{,} \PY{n}{w2}\PY{p}{)}
             \PY{n}{w1\PYZus{}w3\PYZus{}dist} \PY{o}{=} \PY{n}{wv\PYZus{}from\PYZus{}bin}\PY{o}{.}\PY{n}{distance}\PY{p}{(}\PY{n}{w1}\PY{p}{,} \PY{n}{w3}\PY{p}{)}
         
             \PY{n+nb}{print}\PY{p}{(}\PY{l+s+s2}{\PYZdq{}}\PY{l+s+s2}{Synonyms }\PY{l+s+si}{\PYZob{}\PYZcb{}}\PY{l+s+s2}{, }\PY{l+s+si}{\PYZob{}\PYZcb{}}\PY{l+s+s2}{ have cosine distance: }\PY{l+s+si}{\PYZob{}\PYZcb{}}\PY{l+s+s2}{\PYZdq{}}\PY{o}{.}\PY{n}{format}\PY{p}{(}\PY{n}{w1}\PY{p}{,} \PY{n}{w2}\PY{p}{,} \PY{n}{w1\PYZus{}w2\PYZus{}dist}\PY{p}{)}\PY{p}{)}
             \PY{n+nb}{print}\PY{p}{(}\PY{l+s+s2}{\PYZdq{}}\PY{l+s+s2}{Antonyms }\PY{l+s+si}{\PYZob{}\PYZcb{}}\PY{l+s+s2}{, }\PY{l+s+si}{\PYZob{}\PYZcb{}}\PY{l+s+s2}{ have cosine distance: }\PY{l+s+si}{\PYZob{}\PYZcb{}}\PY{l+s+s2}{\PYZdq{}}\PY{o}{.}\PY{n}{format}\PY{p}{(}\PY{n}{w1}\PY{p}{,} \PY{n}{w3}\PY{p}{,} \PY{n}{w1\PYZus{}w3\PYZus{}dist}\PY{p}{)}\PY{p}{)}
         
             \PY{c+c1}{\PYZsh{} \PYZhy{}\PYZhy{}\PYZhy{}\PYZhy{}\PYZhy{}\PYZhy{}\PYZhy{}\PYZhy{}\PYZhy{}\PYZhy{}\PYZhy{}\PYZhy{}\PYZhy{}\PYZhy{}\PYZhy{}\PYZhy{}\PYZhy{}\PYZhy{}}
\end{Verbatim}

    \begin{Verbatim}[commandchars=\\\{\}]
Synonyms good, fine have cosine distance: 0.48414766788482666
Antonyms good, bad have cosine distance: 0.28903740644454956

    \end{Verbatim}

    \hypertarget{write-your-answer-here.}{%
\paragraph{Write your answer here.}\label{write-your-answer-here.}}

Cosine Distance(good,bad) \textless{} Cosine Distance(good,fine) Cosine
Similarity does not look at the magnitude between words, but measures
the angle. It might seem intuitive if the output of w1,w2 was opposite
of w1,w3, rather having close proximity w1,w2(0.484) and w1,w3(0.289).

    \hypertarget{solving-analogies-with-word-vectors}{%
\subsubsection{Solving Analogies with Word
Vectors}\label{solving-analogies-with-word-vectors}}

Word vectors have been shown to \emph{sometimes} exhibit the ability to
solve analogies.

As an example, for the analogy ``man : king :: woman : x'' (read: man is
to king as woman is to x), what is x?

In the cell below, we show you how to use word vectors to find x. The
\texttt{most\_similar} function finds words that are most similar to the
words in the \texttt{positive} list and most dissimilar from the words
in the \texttt{negative} list. The answer to the analogy will be the
word ranked most similar (largest numerical value).

\textbf{Note:} Further Documentation on the \texttt{most\_similar}
function can be found within the
\textbf{\href{https://radimrehurek.com/gensim/models/keyedvectors.html\#gensim.models.keyedvectors.FastTextKeyedVectors.most_similar}{GenSim
documentation}}.

    \begin{Verbatim}[commandchars=\\\{\}]
{\color{incolor}In [{\color{incolor}20}]:} \PY{c+c1}{\PYZsh{} Run this cell to answer the analogy \PYZhy{}\PYZhy{} man : king :: woman : x}
         \PY{n}{pprint}\PY{o}{.}\PY{n}{pprint}\PY{p}{(}\PY{n}{wv\PYZus{}from\PYZus{}bin}\PY{o}{.}\PY{n}{most\PYZus{}similar}\PY{p}{(}\PY{n}{positive}\PY{o}{=}\PY{p}{[}\PY{l+s+s1}{\PYZsq{}}\PY{l+s+s1}{woman}\PY{l+s+s1}{\PYZsq{}}\PY{p}{,} \PY{l+s+s1}{\PYZsq{}}\PY{l+s+s1}{king}\PY{l+s+s1}{\PYZsq{}}\PY{p}{]}\PY{p}{,} \PY{n}{negative}\PY{o}{=}\PY{p}{[}\PY{l+s+s1}{\PYZsq{}}\PY{l+s+s1}{man}\PY{l+s+s1}{\PYZsq{}}\PY{p}{]}\PY{p}{)}\PY{p}{)}
\end{Verbatim}

    \begin{Verbatim}[commandchars=\\\{\}]
[('queen', 0.6978678703308105),
 ('princess', 0.6081745028495789),
 ('monarch', 0.5889754891395569),
 ('throne', 0.5775108933448792),
 ('prince', 0.5750998258590698),
 ('elizabeth', 0.5463595986366272),
 ('daughter', 0.5399125814437866),
 ('kingdom', 0.5318052172660828),
 ('mother', 0.5168544054031372),
 ('crown', 0.5164473056793213)]

    \end{Verbatim}

    \hypertarget{question-2.4-finding-analogies-code-written-2-points}{%
\subsubsection{Question 2.4: Finding Analogies {[}code + written{]} (2
Points)}\label{question-2.4-finding-analogies-code-written-2-points}}

Find an example of analogy that holds according to these vectors
(i.e.~the intended word is ranked top). In your solution please state
the full analogy in the form x:y :: a:b. If you believe the analogy is
complicated, explain why the analogy holds in one or two sentences.

\textbf{Note}: You may have to try many analogies to find one that
works!

    \begin{Verbatim}[commandchars=\\\{\}]
{\color{incolor}In [{\color{incolor}21}]:}     \PY{c+c1}{\PYZsh{} \PYZhy{}\PYZhy{}\PYZhy{}\PYZhy{}\PYZhy{}\PYZhy{}\PYZhy{}\PYZhy{}\PYZhy{}\PYZhy{}\PYZhy{}\PYZhy{}\PYZhy{}\PYZhy{}\PYZhy{}\PYZhy{}\PYZhy{}\PYZhy{}}
             \PY{c+c1}{\PYZsh{} Write your implementation here.}
         \PY{n+nb}{print}\PY{p}{(}\PY{l+s+s1}{\PYZsq{}}\PY{l+s+s1}{\PYZsq{}}\PY{p}{)}
         \PY{n+nb}{print}\PY{p}{(}\PY{l+s+s1}{\PYZsq{}}\PY{l+s+s1}{white : black :: up : down}\PY{l+s+s1}{\PYZsq{}}\PY{p}{)}
         \PY{n+nb}{print}\PY{p}{(}\PY{l+s+s1}{\PYZsq{}}\PY{l+s+s1}{\PYZsq{}}\PY{p}{)}
         \PY{n}{pprint}\PY{o}{.}\PY{n}{pprint}\PY{p}{(}\PY{n}{wv\PYZus{}from\PYZus{}bin}\PY{o}{.}\PY{n}{most\PYZus{}similar}\PY{p}{(}\PY{n}{positive}\PY{o}{=}\PY{p}{[}\PY{l+s+s1}{\PYZsq{}}\PY{l+s+s1}{white}\PY{l+s+s1}{\PYZsq{}}\PY{p}{,} \PY{l+s+s1}{\PYZsq{}}\PY{l+s+s1}{up}\PY{l+s+s1}{\PYZsq{}}\PY{p}{]}\PY{p}{,} \PY{n}{negative}\PY{o}{=}\PY{p}{[}\PY{l+s+s1}{\PYZsq{}}\PY{l+s+s1}{down}\PY{l+s+s1}{\PYZsq{}}\PY{p}{]}\PY{p}{)}\PY{p}{)}
         
         
             \PY{c+c1}{\PYZsh{} \PYZhy{}\PYZhy{}\PYZhy{}\PYZhy{}\PYZhy{}\PYZhy{}\PYZhy{}\PYZhy{}\PYZhy{}\PYZhy{}\PYZhy{}\PYZhy{}\PYZhy{}\PYZhy{}\PYZhy{}\PYZhy{}\PYZhy{}\PYZhy{}}
\end{Verbatim}

    \begin{Verbatim}[commandchars=\\\{\}]

white : black :: up : down

[('black', 0.709771990776062),
 ('with', 0.5913417339324951),
 ('red', 0.5865756869316101),
 ('brown', 0.5703251361846924),
 ('green', 0.552123486995697),
 ('colored', 0.5432795286178589),
 ('yellow', 0.531940221786499),
 ('blue', 0.5310556888580322),
 ('pink', 0.5272434949874878),
 ('house', 0.5265994071960449)]

    \end{Verbatim}

    \hypertarget{write-your-answer-here.}{%
\paragraph{Write your answer here.}\label{write-your-answer-here.}}

`white : black :: up : down'. Black was produced and is the correct
analogy.

    \hypertarget{question-2.5-incorrect-analogy-code-written-1-point}{%
\subsubsection{Question 2.5: Incorrect Analogy {[}code + written{]} (1
point)}\label{question-2.5-incorrect-analogy-code-written-1-point}}

Find an example of analogy that does \emph{not} hold according to these
vectors. In your solution, state the intended analogy in the form x:y ::
a:b, and state the (incorrect) value of b according to the word vectors.

    \begin{Verbatim}[commandchars=\\\{\}]
{\color{incolor}In [{\color{incolor}22}]:}     \PY{c+c1}{\PYZsh{} \PYZhy{}\PYZhy{}\PYZhy{}\PYZhy{}\PYZhy{}\PYZhy{}\PYZhy{}\PYZhy{}\PYZhy{}\PYZhy{}\PYZhy{}\PYZhy{}\PYZhy{}\PYZhy{}\PYZhy{}\PYZhy{}\PYZhy{}\PYZhy{}}
             \PY{c+c1}{\PYZsh{} Write your implementation here.}
         
         \PY{n+nb}{print}\PY{p}{(}\PY{l+s+s1}{\PYZsq{}}\PY{l+s+s1}{\PYZsq{}}\PY{p}{)}
         \PY{n+nb}{print}\PY{p}{(}\PY{l+s+s1}{\PYZsq{}}\PY{l+s+s1}{Dog : puppy :: cat : kitten}\PY{l+s+s1}{\PYZsq{}}\PY{p}{)}
         \PY{n+nb}{print}\PY{p}{(}\PY{l+s+s1}{\PYZsq{}}\PY{l+s+s1}{\PYZsq{}}\PY{p}{)}
         \PY{n}{pprint}\PY{o}{.}\PY{n}{pprint}\PY{p}{(}\PY{n}{wv\PYZus{}from\PYZus{}bin}\PY{o}{.}\PY{n}{most\PYZus{}similar}\PY{p}{(}\PY{n}{positive}\PY{o}{=}\PY{p}{[}\PY{l+s+s1}{\PYZsq{}}\PY{l+s+s1}{cat}\PY{l+s+s1}{\PYZsq{}}\PY{p}{,} \PY{l+s+s1}{\PYZsq{}}\PY{l+s+s1}{dog}\PY{l+s+s1}{\PYZsq{}}\PY{p}{]}\PY{p}{,} \PY{n}{negative}\PY{o}{=}\PY{p}{[}\PY{l+s+s1}{\PYZsq{}}\PY{l+s+s1}{puppy}\PY{l+s+s1}{\PYZsq{}}\PY{p}{]}\PY{p}{)}\PY{p}{)} 
         
             \PY{c+c1}{\PYZsh{} \PYZhy{}\PYZhy{}\PYZhy{}\PYZhy{}\PYZhy{}\PYZhy{}\PYZhy{}\PYZhy{}\PYZhy{}\PYZhy{}\PYZhy{}\PYZhy{}\PYZhy{}\PYZhy{}\PYZhy{}\PYZhy{}\PYZhy{}\PYZhy{}}
\end{Verbatim}

    \begin{Verbatim}[commandchars=\\\{\}]

Dog : puppy :: cat : kitten

[('dogs', 0.6272706389427185),
 ('cats', 0.5924872756004333),
 ('horse', 0.5564484596252441),
 ('pet', 0.5380032062530518),
 ('animal', 0.5019071102142334),
 ('rat', 0.4833652973175049),
 ('bird', 0.47245192527770996),
 ('eye', 0.47093015909194946),
 ('bear', 0.4681668281555176),
 ('animals', 0.4641147553920746)]

    \end{Verbatim}

    \hypertarget{write-your-answer-here.}{%
\paragraph{Write your answer here.}\label{write-your-answer-here.}}

Dog : puppy :: cat : kitten. Kitten was the correct analogy, however it
was not produced.

    \hypertarget{question-2.6-guided-analysis-of-bias-in-word-vectors-written-1-point}{%
\subsubsection{Question 2.6: Guided Analysis of Bias in Word Vectors
{[}written{]} (1
point)}\label{question-2.6-guided-analysis-of-bias-in-word-vectors-written-1-point}}

It's important to be cognizant of the biases (gender, race, sexual
orientation etc.) implicit in our word embeddings. Bias can be dangerous
because it can reinforce stereotypes through applications that employ
these models.

Run the cell below, to examine (a) which terms are most similar to
``woman'' and ``worker'' and most dissimilar to ``man'', and (b) which
terms are most similar to ``man'' and ``worker'' and most dissimilar to
``woman''. Point out the difference between the list of
female-associated words and the list of male-associated words, and
explain how it is reflecting gender bias.

    \begin{Verbatim}[commandchars=\\\{\}]
{\color{incolor}In [{\color{incolor}23}]:} \PY{c+c1}{\PYZsh{} Run this cell}
         \PY{c+c1}{\PYZsh{} Here `positive` indicates the list of words to be similar to and `negative` indicates the list of words to be}
         \PY{c+c1}{\PYZsh{} most dissimilar from.}
         \PY{n}{pprint}\PY{o}{.}\PY{n}{pprint}\PY{p}{(}\PY{n}{wv\PYZus{}from\PYZus{}bin}\PY{o}{.}\PY{n}{most\PYZus{}similar}\PY{p}{(}\PY{n}{positive}\PY{o}{=}\PY{p}{[}\PY{l+s+s1}{\PYZsq{}}\PY{l+s+s1}{woman}\PY{l+s+s1}{\PYZsq{}}\PY{p}{,} \PY{l+s+s1}{\PYZsq{}}\PY{l+s+s1}{worker}\PY{l+s+s1}{\PYZsq{}}\PY{p}{]}\PY{p}{,} \PY{n}{negative}\PY{o}{=}\PY{p}{[}\PY{l+s+s1}{\PYZsq{}}\PY{l+s+s1}{man}\PY{l+s+s1}{\PYZsq{}}\PY{p}{]}\PY{p}{)}\PY{p}{)}
         \PY{n+nb}{print}\PY{p}{(}\PY{p}{)}
         \PY{n}{pprint}\PY{o}{.}\PY{n}{pprint}\PY{p}{(}\PY{n}{wv\PYZus{}from\PYZus{}bin}\PY{o}{.}\PY{n}{most\PYZus{}similar}\PY{p}{(}\PY{n}{positive}\PY{o}{=}\PY{p}{[}\PY{l+s+s1}{\PYZsq{}}\PY{l+s+s1}{man}\PY{l+s+s1}{\PYZsq{}}\PY{p}{,} \PY{l+s+s1}{\PYZsq{}}\PY{l+s+s1}{worker}\PY{l+s+s1}{\PYZsq{}}\PY{p}{]}\PY{p}{,} \PY{n}{negative}\PY{o}{=}\PY{p}{[}\PY{l+s+s1}{\PYZsq{}}\PY{l+s+s1}{woman}\PY{l+s+s1}{\PYZsq{}}\PY{p}{]}\PY{p}{)}\PY{p}{)}
\end{Verbatim}

    \begin{Verbatim}[commandchars=\\\{\}]
[('employee', 0.6375863552093506),
 ('workers', 0.6068919897079468),
 ('nurse', 0.5837947130203247),
 ('pregnant', 0.5363885760307312),
 ('mother', 0.5321309566497803),
 ('employer', 0.5127025842666626),
 ('teacher', 0.5099577307701111),
 ('child', 0.5096741914749146),
 ('homemaker', 0.5019455552101135),
 ('nurses', 0.4970571994781494)]

[('workers', 0.611325740814209),
 ('employee', 0.5983108878135681),
 ('working', 0.5615329742431641),
 ('laborer', 0.5442320108413696),
 ('unemployed', 0.5368517637252808),
 ('job', 0.5278826951980591),
 ('work', 0.5223963260650635),
 ('mechanic', 0.5088937282562256),
 ('worked', 0.5054520964622498),
 ('factory', 0.4940453767776489)]

    \end{Verbatim}

    \hypertarget{write-your-answer-here.}{%
\paragraph{Write your answer here.}\label{write-your-answer-here.}}

`Workers' is a gender neutral term, however in the first list men can be
nurses, employees, homemaker, etc.. In the second list, woman can be
workers, employees, working, laborers, etc.

    \hypertarget{question-2.7-independent-analysis-of-bias-in-word-vectors-code-written-1-point}{%
\subsubsection{Question 2.7: Independent Analysis of Bias in Word
Vectors {[}code + written{]} (1
point)}\label{question-2.7-independent-analysis-of-bias-in-word-vectors-code-written-1-point}}

Use the \texttt{most\_similar} function to find another case where some
bias is exhibited by the vectors. Please briefly explain the example of
bias that you discover.

    \begin{Verbatim}[commandchars=\\\{\}]
{\color{incolor}In [{\color{incolor}24}]:}     \PY{c+c1}{\PYZsh{} \PYZhy{}\PYZhy{}\PYZhy{}\PYZhy{}\PYZhy{}\PYZhy{}\PYZhy{}\PYZhy{}\PYZhy{}\PYZhy{}\PYZhy{}\PYZhy{}\PYZhy{}\PYZhy{}\PYZhy{}\PYZhy{}\PYZhy{}\PYZhy{}}
             \PY{c+c1}{\PYZsh{} Write your implementation here.}
         
         \PY{n}{pprint}\PY{o}{.}\PY{n}{pprint}\PY{p}{(}\PY{n}{wv\PYZus{}from\PYZus{}bin}\PY{o}{.}\PY{n}{most\PYZus{}similar}\PY{p}{(}\PY{n}{positive}\PY{o}{=}\PY{p}{[}\PY{l+s+s1}{\PYZsq{}}\PY{l+s+s1}{man}\PY{l+s+s1}{\PYZsq{}}\PY{p}{,} \PY{l+s+s1}{\PYZsq{}}\PY{l+s+s1}{solider}\PY{l+s+s1}{\PYZsq{}}\PY{p}{]}\PY{p}{,} \PY{n}{negative}\PY{o}{=}\PY{p}{[}\PY{l+s+s1}{\PYZsq{}}\PY{l+s+s1}{woman}\PY{l+s+s1}{\PYZsq{}}\PY{p}{]}\PY{p}{)}\PY{p}{)}
         \PY{n+nb}{print}\PY{p}{(}\PY{p}{)}
         \PY{n}{pprint}\PY{o}{.}\PY{n}{pprint}\PY{p}{(}\PY{n}{wv\PYZus{}from\PYZus{}bin}\PY{o}{.}\PY{n}{most\PYZus{}similar}\PY{p}{(}\PY{n}{positive}\PY{o}{=}\PY{p}{[}\PY{l+s+s1}{\PYZsq{}}\PY{l+s+s1}{woman}\PY{l+s+s1}{\PYZsq{}}\PY{p}{,} \PY{l+s+s1}{\PYZsq{}}\PY{l+s+s1}{solider}\PY{l+s+s1}{\PYZsq{}}\PY{p}{]}\PY{p}{,} \PY{n}{negative}\PY{o}{=}\PY{p}{[}\PY{l+s+s1}{\PYZsq{}}\PY{l+s+s1}{man}\PY{l+s+s1}{\PYZsq{}}\PY{p}{]}\PY{p}{)}\PY{p}{)}
         
             \PY{c+c1}{\PYZsh{} \PYZhy{}\PYZhy{}\PYZhy{}\PYZhy{}\PYZhy{}\PYZhy{}\PYZhy{}\PYZhy{}\PYZhy{}\PYZhy{}\PYZhy{}\PYZhy{}\PYZhy{}\PYZhy{}\PYZhy{}\PYZhy{}\PYZhy{}\PYZhy{}}
\end{Verbatim}

    \begin{Verbatim}[commandchars=\\\{\}]
[('serviceman', 0.47422173619270325),
 ('soldier', 0.453141987323761),
 ('soliders', 0.4464874863624573),
 ('peacekeeper', 0.4324309229850769),
 ('beefy', 0.4253820478916168),
 ('crewman', 0.4212471842765808),
 ('mustached', 0.3806219696998596),
 ('privates', 0.3796985149383545),
 ('sgt', 0.3769349455833435),
 ('mutineer', 0.3749666213989258)]

[('soliders', 0.5773683786392212),
 ('serviceman', 0.5757607817649841),
 ('solders', 0.49688705801963806),
 ('passerby', 0.48241591453552246),
 ('servicemember', 0.46611881256103516),
 ('crewmember', 0.4597702920436859),
 ('schoolmistress', 0.4408072233200073),
 ('peacekeeper', 0.44013288617134094),
 ('policewoman', 0.4347541034221649),
 ('bystander', 0.4280003011226654)]

    \end{Verbatim}

    \hypertarget{write-your-answer-here.}{%
\paragraph{Write your answer here.}\label{write-your-answer-here.}}

The number of women `soldiers' (servicemen and servicewoman) have
increased over the years within the army. In today's terminology a
solider is either a man or woman and acccording to the output `solider'
is closer to `man' than `woman'.

    \hypertarget{question-2.8-thinking-about-bias-written-2-points}{%
\subsubsection{Question 2.8: Thinking About Bias {[}written{]} (2
points)}\label{question-2.8-thinking-about-bias-written-2-points}}

What might be the causes of these biases in the word vectors? You should
give least 2 explainations how bias get into the word vectors. How might
you be able to investigate/test these causes?

    \hypertarget{write-your-answer-here.}{%
\paragraph{Write your answer here.}\label{write-your-answer-here.}}

Female terms such as ``her'', ``she'', ``woman'' are closer to family
and arts/education than they are to professional positions and math
studies, whereas the reverse is true for male terms. Vector embeddings
used on older content could capture these biases in their relationships.
Another way biases can get into word vectors is that patterns can exist
between people's names and common lower-case words and phrases, even
more so in other languages like German (`der' for the masculine nouns,
`die' for female.

Testing for bias can include searching the original training documents
and removing biases and retraining algorthims used word embeddings.

    \hypertarget{submission-instructions}{%
\section{ Submission Instructions}\label{submission-instructions}}

\begin{enumerate}
\def\labelenumi{\arabic{enumi}.}
\tightlist
\item
  Click the Save button at the top of the Jupyter Notebook.
\item
  Select Cell -\textgreater{} All Output -\textgreater{} Clear. This
  will clear all the outputs from all cells (but will keep the content
  of all cells).
\item
  Select Cell -\textgreater{} Run All. This will run all the cells in
  order, and will take several minutes.
\item
  Once you've rerun everything, select File -\textgreater{} Download as
  -\textgreater{} PDF via LaTeX (If you have trouble using ``PDF via
  LaTex'', you can also save the webpage as pdf. Make sure all your
  solutions especially the coding parts are displayed in the pdf, it's
  okay if the provided codes get cut off because lines are not wrapped
  in code cells).
\item
  Look at the PDF file and make sure all your solutions are there,
  displayed correctly. The PDF is the only thing your graders will see!
\item
  Submit your PDF on Gradescope.
\end{enumerate}

    \begin{Verbatim}[commandchars=\\\{\}]
{\color{incolor}In [{\color{incolor} }]:} 
\end{Verbatim}


    % Add a bibliography block to the postdoc
    
    
    
    \end{document}
